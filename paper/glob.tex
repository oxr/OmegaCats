\section{Globular Sets}


In Type Theory we use the notion of a setoid to describe a set with a
specific equality. That is a $A : \Setoid$ is given by the
following data:
\begin{eqnarray*}
  \obj_A & : & \Set \\
  \hom_A & : & \obj_A \rightarrow \obj_A \rightarrow \Prop
\end{eqnarray*}
and proof objects $\id, \sym{-}, \trans{-}{-}$ witnessing that $\hom$ is an
equivalence relation. Here we write $\Prop$ to denote the class of sets
which have at most one inhabitant. This restriction is important when
showing that the category of setoids has certain structure, in
particular forms a model of Type Theory. That is setoids can model a
type theory with a proof-irrelevant equality. To model proof-relevant
equality we need to generalize the notion of a setoid so that the
$\hom$-sets are generalized setoids again. It is not enough to just
postulate the laws of an equivalence relation at each level, we also
need some laws how these proofs interact. On the first level we
require the laws of a groupoid, e.g. we want that $\trans{\id}{\alpha}$ is
equal to $\alpha$. Here \emph{is equal} means that they are related by the
equality relation of a setoid again. Clearly, the structure we are
looking for is a weak $\omega$-groupoid. It is the goal of this paper
to develop a formalisation of this structure. As a first step let's
ignore the proof objects (i.e. the data of an equivalence relation and
the goupoid laws etc).
We end up with a coinductive definition of a \emph{globular set} $G :
\Glob$ given by
\begin{eqnarray*}
  \obj_G & : & \Set \\
  \hom_G & : & \obj_G \rightarrow \obj_G \rightarrow \infty\,\Glob
\end{eqnarray*}
Here we use $\infty$ to indicate that $\Glob$ is defined
coinductively. More formally, $\Glob$ is the terminal coalgebra of
$\Sigma \obj : \Set. \obj \rightarrow \obj \rightarrow -$. Given globular sets
$A,B$ a morphism $f : \Glob(A , B)$ between them is given by 
\begin{eqnarray*}
  \objto_f & : & \obj_A \rightarrow \obj_B \\
  \homto_f & : & \Pi a,b:\obj_A. \Glob(\hom_A\,a\,b,\hom_B(\objto_f\,a,\objto_f\,b))
\end{eqnarray*}
Note that this definition exploits the coinductive character of
$\Glob$. Identity and composition can be defined easily by iterating
the set-theoretic definitions ad infinitum. As an example we can
define the terminal object in $\Unit : \Glob$ by the equations
\begin{eqnarray*}
  \obj_\Unit & = & \Unit_\Set \\
  \hom_\Unit\,x\,y & = & \Unit
\end{eqnarray*}
More interestingly, the globular set of identity proofs over a given
set $A$, $\Idw\,A : \Glob$ can be defined as follows:
\begin{eqnarray*}
\obj_{\Idw\,A} & = & A \\
\hom_{\Idw\,A}\,a\,b & = & \Idw\,(a = b)
\end{eqnarray*}
Our definition of globular sets is equivalent to the usual one as a
presheaf category over the diagram:
\[
0 \two^{s_0}_{t_0} 1 \two^{s_1}_{t_1} 2 \dots n \two^{s_n}_{t_n} (n+1) \dots  
\]
with the globular identities:
\begin{eqnarray*}
  t_{i+1} \circ s_i & = & s_{i+1} \circ t_i \\
  t_{i+1} \circ t_i & = & s_{i+1} \circ t_i
\end{eqnarray*}
In the example of $\Idw\,A$ the presheaf is given by a family
$F^A : \Nat \rightarrow \Set$:
\begin{eqnarray*}
  F^A\,0 & = & A\\
  F^A\,1 & = & \Sigma a,b : A,a = b\\
  F^A\,2 & = & \Sigma a,b : A,\Sigma \alpha,\beta: a = b, \alpha = \beta\\
  \vdots & \vdots & \vdots \\
  F^A\,(n+1) & = & \Sigma a,b:A,F^{a = b}\,n
\end{eqnarray*}
and source and target maps $s_i,t_I  F^A\,(n+1) \rightarrow F^A\,n$ satisfying the globular identities. 
\begin{eqnarray*}
s_0 (a,b,-) & = & a \\
t_0 (a,b,-) & = & b \\  
s_{n+1}\,(a,b,\alpha)  & = & (a , b, s_n\,\alpha) \\
t_{n+1}\,(a,b,\alpha)  & = & (a , b, t_n\,\alpha) \\
\end{eqnarray*}

It follows from the results
\cite{lumsdaine10:weak-o-categories,berg08:types-are} that the eliminator $J$ is sufficent
to establish that $\Idw\,A$ is a weak $\omega$ groupoid. In their work
this is done on the meta level reasoning about models of type
theory. In the present paper our goal is to show that this can be done
inside Type Theory.

%%% Local Variables:
%%% TeX-master: "weakomega2"
%%% End: