
\section{Laws}
\label{sec:laws}

\begin{quote}
  \begin{itemize}
  \item telescope morphisms

  \item inverse

  \item lambda,rho,alpha,interchange,inverse-laws

  \end{itemize}
\end{quote}

%
In a strict $\omega$-category composition and
identities -- structure -- are accompanied by axioms expressing their fundamental
properties. Namely, composition should be \emph{associative}, and it
should satisfy the so-called \emph{interchange law}; identities should
be the \emph{units} of composition. The fact that the axioms are
equations and one can therefore replace equals for equals in
expressions has the pleasant consequence the
complexity of axioms doesn't increase with dimension. Indeed, the whole theory
for strict $\omega$-categories can be generalised without much difficulty
to categories enriched in an arbitrary monoidal
category \cite{kelly:1982}. However, once equations are cateogorified,
i.e. when equations are replaced by data -- \emph{coherence cells} -- their
complexity rises steeply with dimension. 

For example, the 1-categorical left-unity law: 
\begin{equation}\label{eq:lambda-f-strictly}
\id_b \circ f \quad = \quad f \enspace,
\end{equation}
 for all $f : a \longrightarrow b$, is replaced in a weak
$\omega$-category by a pair of 2-cells $\lambda_f: \id_b \circ f
\Longrightarrow f $ and $\lambda^{-1}_f : f \Longrightarrow \id_b \circ
f$. A similar law should hold for $\circ$ and higher cells. I.e. it
should also hold -- strictly -- that for any $\alpha: f
\Longrightarrow f'$: 
\begin{equation}\label{eq:lambda-alpha-strictly}
\id^2_b \circ \alpha \quad = \quad \alpha\enspace,
\end{equation}
where
$\id^2_b = \id_{\id_b}$. Note that \eqref{eq:lambda-alpha-strictly}
makes sense because \eqref{eq:lambda-f-strictly} holds. In the
weak case, it is not the case that the boundary of $\id^2\circ\alpha$
is the same as the boundary of $\alpha$ and it is simply not possible
to categorify \eqref{eq:lambda-alpha-strictly} by introducing a pair
of 3-cells. However, we can use $\lambda_f$ and $\lambda^{-1}_f$
to coerce $\id_b \circ f$ to $f$ and $\id_b \circ f'$ to $f'$ making
the left and right-hand sides agree. The following figure illustrates
this:
\begin{equation}\label{eq:lambda-alpha-morphism}
\lambda_\alpha \; : \;
\bfig
\morphism/{@{>}@/^1em/}/[a`b;f]
\morphism(500,0)/{@{>}@/^1em/}/[b`b;\id\,f]
\morphism/{@{>}@/^3em/}/<1000,0>[a`b;f]
\morphism(500,250)|r|/=>/<0,-100>[`;\lambda^{-1}_f]
\morphism|b|/{@{>}@/_1em/}/[a`b;f']
\morphism(500,0)|b|/{@{>}@/_1em/}/[b`b;\id\,b]
\morphism(250,50)/=>/<0,-100>[`;\alpha]
\morphism(750,50)/=>/<0,-100>[`;\id^2 b]
\morphism|b|/{@{>}@/_3em/}/<1000,0>[a`b;f']
\morphism(500,-100)|r|/=>/<0,-100>[`;\lambda_{f'}]
\efig
\quad\Rrightarrow\quad
\bfig\
\morphism/{@{>}@/^2em/}/<500,0>[a`b;f]
\morphism(250,50)/=>/<0,-100>[`;\alpha]
\morphism|b|/{@{>}@/_2em/}/<500,0>[a`b;f']
\efig
\end{equation}
% 
The reader is invited to try to write down the fourth iteration,
i.e. the domain and codomain of $\lambda_\gamma$ for $\gamma : \alpha
\Rrightarrow \alpha' : f \Longrightarrow 'f : a \longrightarrow
a'$. Note that each higher pair of $\lambda$'s can be seen as
expressing the naturality of the preceding lower lambda. 

Similarly one must introduce $\rho$'s to witness the right unit law,
$\chi$'s to witness interchange, and $\alpha$'s to witness
associativity. Note that we have chosen the example of $\lambda$'s
because of its relative simplicity. 
%
Moreover, all such coherence cells must satisfy a coherence property
basically saying that any pair of $n$-cells from $d$ to $d'$ involving
only coherence cells\footnote{identity cells can be seen as witnessing
  reflexivity of equality} must have a mediating $n+1$-cell connecting
$d$ and $d'$. Intuitively, as coherence cells are witnesses of
equality proofs the higher coherence cells are witnessing uniqueness of
equality proofs.




This combinatorial complexity of coherence cells has been the major
obstacle in the development of weak $\omega$-categories. This has led
to the development of many diverse approaches to weak
$\omega$-categories \cite{}. However the development of Type Theory
has made it possible to express all coherence cells in a closed
form. In this section we start to describe how in Type Theory all
coherence cells can be generated by induction on their dimension.

\subsection{Formalising left units}
\label{sec:lambdas}

In \eqref{eq:lambda-alpha-morphism} we made the boundaries of the left- and right-hand
sides match by applying the function:
\[
\Phi \quad \equiv \quad (l,l') \quad \mapsto \quad x\quad
\mapsto \quad l \cdot x \cdot l' \] 
%
to $(\lambda^{-1}_f, \lambda_{f'})$ and $\id^k\,b \circ \alpha )$, where
$k$ is the depth of the telescope $\alpha$ lies in. The 3-cells
$\lambda_\alpha$ and $\lambda^{-1}_\alpha$ are then introduced as
%
\[\begin{array}{ll}\lambda_\alpha & : \quad
  \Obj~(\homcat{\homcat{\homcat{\bullet}{a}{b}}{f}{f'}}{\Phi\,(\lambda^{-1}_f,\lambda_{f'}) \,
    (\id^2\,b \circ \alpha) }{\alpha})\\
\lambda^{-1}_\alpha & : \quad \Obj~(\homcat{\homcat{\homcat{\bullet}{a}{b}}{f}{f'}}{\alpha}{\Phi\,(
 \lambda^{-1}_f,\lambda_{f'})\,(\id^2\,b\circ \alpha)})
\end{array}
\enspace.\] 
%
In a diagram:
\[
\bfig
\square/<-`>`>`>/<750,500>[ \id\,b \circ f ` f  ` \id\,b \circ
f' ` f';\lambda^{-1}_f`(\id^2\,b)\,\circ\,\alpha`\alpha`\lambda_{f'}]
\morphism(300,180)/@3{->}/<200,0>[`;\lambda_\alpha]
\morphism(300,320)/@3{<-}/<200,0>[`;\lambda^{-1}_\alpha]
\efig
\]
%
These arrows should be natural in 3-cells $\gamma : \alpha \Rrightarrow
\alpha'$ in the following sense: 
\[
\bfig
\square/<-`>`>`>/<1000,750>[ \lambda_{f'}\ast (\id^2\,b \circ \alpha)
\ast \lambda^{-1}_f ` \alpha  ` \lambda_{f'}\ast(\id^2\,b \circ
\alpha')\ast \lambda^{-1}_f ` \alpha';\lambda^{-1}_\alpha` \id\,{\lambda_{f'}}\ast((\id^3\,b)\,\circ\,\gamma)\ast \id\,{\lambda^{-1}_f}`\gamma`\lambda_{\alpha'}]
\morphism(350,300)/@3{->}/<200,0>[`;\lambda_\gamma]
\morphism(350,440)/@3{<-}/<200,0>[`;\lambda^{-1}_\gamma]
\efig
\]
%
Now note that going top-left-bottom  around the square one gets
%
\[\Phi\,(\lambda^{-1}_\alpha, \lambda_{\alpha'})\,
(\Phi\,(\lambda^{-1}_f, \lambda_{f'}) \, \gamma))\enspace.\]
%
This looks to be a good basis for the recursion generating all higher
$\lambda$'s. A similar pattern occurs in the definition of the other
coherence cells.


\subsection{Formalising all coherence cells}\label{sec:formalising-coherence}
To summarise and generalise the previous section, we want to introduce
for each $\alpha$ in a telescope $t$ of depth $n$ and each $\beta$ in
a telescope $u$ of depth $n$ a cell $\Phi\,m \,\alpha \longrightarrow
\beta$ where $m$ is the data necessary to define a function
$\Obj~(\cat{t}) \rightarrow \Obj~(\cat{u})$. We will call such an $m$
a \emph{telescope morphism form $t$ to $u$}; formally $ m : t
\rightleftharpoons u$. Then $\Phi$ has type $t \rightleftharpoons u
\rightarrow \Obj (\cat t) \rightarrow \Obj(\cat u)$

% ------------------------------

% making them unital with respect to
% composition. In the weak setting such laws are replaced by coherence
% cells, which we discuss next.

% With the development of the previous two sections we can express
% compositions such as:
% \[
% \id\,b \circ f \qquad \id^2\,b \circ \alpha\qquad \id\,f'\circ \alpha
% \]
% pictured respectively from left to right as
% \[
% \bfig
% \morphism[a`b;f]
% \morphism(500,0)<250,0>[b`b;\id\,f]
% \efig
% \qquad
% \bfig
% \morphism/{@{>}@/^1em/}/[a`b;f]
% \morphism|b|/{@{>}@/_1em/}/[a`b;f']
% \morphism(250,50)<0,-100>[`;\alpha]
% \morphism(500,0)/{@{>}@/^1em/}/[b`b;\id\,b]
% \morphism(500,0)|b|/{@{>}@/_1em/}/[b`b;\id\,b]
% \morphism(750,50)<0,-100>[`;\id^2 b]
% \efig
% \]\[
% \bfig\scalefactor{1.2}
% \morphism/{@{>}@/^1.7em/}/[a`b;f]
% \morphism|-|[a`b;f']
% \morphism(250,110)<0,-80>[`;\alpha]
% %
% \morphism|b|/{@{>}@/_1.7em/}/[a`b;f']
% \morphism(250,-20)<0,-80>[`;\id\,f']
% \efig
% \enspace.\]
% We want to define the cells that connect these cells to the cells:
% \[
% \bfig
% \morphism[a`b;f]
% \efig
% \qquad
% \bfig
% \morphism/{@{>}@/^1em/}/[a`b;f]
% \morphism|b|/{@{>}@/_1em/}/[a`b;f']
% \morphism(250,50)<0,-100>[`;\alpha]
% \efig
% \qquad
% \bfig
% \morphism/{@{>}@/^1em/}/[a`b;f]
% \morphism|b|/{@{>}@/_1em/}/[a`b;f']
% \morphism(250,50)<0,-100>[`;\alpha]
% \efig
% \]

% Let us analyse the situation at hand.

% Firstly, note that the right-most case is just the left-most case in
% the category $\homcat{\bullet}{a}{b}$ so from now on we assume
% everything is taking place in an arbitrary \emph{base category},
% $\cC$.

% For the left-most case, assuming $a$ and $b$ are 0-cells of
%   $\cC$, we simply introduce the syntax for a 2-cell $\lambda_f$ of $\cC$:
% \[\bfig
% \qtriangle[a`b`b;f`f`\id\,b]
% \morphism(400,400)/<=/<-100,-100>[`;\lambda_f]
% \efig\]

% Having defined a 2-cell $\lambda_f$ for
%   every 1-cell $f$, one can define a 3-cell $\lambda_\alpha$ for every
%   2-cell $\alpha : f \Rightarrow f'$ as going from the left-hand side
%   to the right-hand side below:
% \[\bfig
% \morphism/{@{>}@/^1em/}/[a`b;f]
% \morphism(500,0)/{@{>}@/^1em/}/[b`b;\id\,f]
% \morphism/{@{>}@/^3em/}/<1000,0>[a`b;f]
% \morphism(500,250)/=>/<0,-100>[`;\lambda_f]
% \morphism|b|/{@{>}@/_1em/}/[a`b;f']
% \morphism(500,0)|b|/{@{>}@/_1em/}/[b`b;\id\,b]
% \morphism(250,50)/=>/<0,-100>[`;\alpha]
% \morphism(750,50)/=>/<0,-100>[`;\id^2 b]
% \efig
% \quad\Rrightarrow\quad
% \bfig
% \morphism/{@{>}@/^3em/}/<1000,0>[a`b;f]
% \morphism|m|/{@{>}@/^1.6em/}/<1000,0>[a`b;f']
% \morphism(500,290)/=>/<0,-100>[`;\alpha]
% \morphism|b|/{@{>}@/_1em/}/[a`b;f']
% \morphism(500,0)|b|/{@{>}@/_1em/}/[b`b;\id\,b]
% \morphism(750,90)/=>/<0,-100>[`;\lambda_{f'}]
% \efig
% \]
% Note however that because we are not working in a strict setting
% the pasting together of the left-hand side is not unique. We pick a
% particular one:
% \[
% \lambda_\alpha ~:~ ((\id^2 b) \circ \alpha) \cdot \lambda_f \Rrightarrow \lambda_{f'}\cdot
% \alpha\enspace,
% \]
% where $\circ$ denotes composition in $\cC$ and $\cdot$ denotes
% composition in $\homcat{\cC}{a}{b}$. In $\homcat{\cC}{a}{b}$ this is
% the diagonal filler in the square
% \begin{equation}\label{eq:lambda-natur}
% \bfig
% \Square[f`(\id\,b) \circ f` f'`(\id\,b)\circ f';\lambda_f`\alpha`(\id^2
% b) \circ \alpha`\lambda_{f'}]
% \morphism(500,300)/=>/<-200,-150>[`;\lambda_\alpha]
% \efig
% \end{equation}

% This diagram expresses the naturality of the assignment $f \mapsto
% \lambda_f$ with respect to cells $\alpha : f \Rightarrow f'$ 
% witnessed by $\lambda_\alpha$. 
% %

% When we put:
% \[
% \AxiomC{$t : \Tel~\homcat{C}{a}{b}~n$}
% \AxiomC{$x : \Obj~(t\Downarrow)$}
% \BinaryInfC{$\dom\lambda_0~x~=~x \qquad
%   \cod\lambda_0~x~=~\id^n(\id~b)\circ x$}
% \DisplayProof
% \]
% \[
% \AxiomC{$t : \Tel~\homcat{C}{a}{b}~n$}
% \AxiomC{$x : \Obj~(t\Downarrow)$}
% \BinaryInfC{$\lambda_n x : \Obj~(\telsuc{t}{\dom\lambda_n\, x
%   }{\cod\lambda_n\,x}\Downarrow)$}
% \DisplayProof
% \]
% then \eqref{eq:lambda-natur} can be rewritten as follows:
% \begin{equation}\label{eq:lambda-natur2}
% \bfig
% \Square[\dom\lambda_0\,f`\cod\lambda_0\,f`\dom\lambda_0~f'`\cod\lambda_0~f';
% \lambda_0 f`\dom\lambda_0~\alpha`\cod\lambda_0~\alpha`\lambda_0 f']
% \morphism(500,300)/=>/<-200,-150>[`;\lambda_1 \alpha]
% \efig\enspace,
% \end{equation}
% where necessarily 
% \[
% \AxiomC{$\dom\lambda_1\,\alpha = \cod\lambda_0\,\alpha\cdot
%   \lambda_0\,f$}
% \DisplayProof
% \quad \AxiomC{$\cod\lambda_1\,\alpha = \lambda_0\, f'\cdot
%   \dom\lambda_0\,\alpha$}
% \DisplayProof
% \]
% Here $\cdot$ denotes vertical composition of 2-cells and $\circ$ is,
% as before, horizontal composition.

% %
% To generalise and iterate \eqref{eq:lambda-natur2} we
% introduce an $n+1$ cell $\quad\lambda~t~x\quad$ for each telescope
% $\quad t :\Tel~\homcat{C}{a}{b}~n\quad$ and $\quad x :
% \Obj~(t\Downarrow)\quad$. At the same time we introduce functions
% $\dom\lambda$ and $\cod\lambda$ taking a telescope $t$, a telescope
% $u$ relative to $t$, formally $u : \Tel ~ (t \Downarrow)~ m$ and a
% cell $x : \Obj (u\Downarrow)$ to a new telescope and a cell in it so
% that the following makes sense:
% \begin{equation}\label{eq:lambda-natur3}
% \bfig
% \Square|ammb|[\dom\lambda~t~\telzero{t\Downarrow}~x`
% \cod\lambda~t~\telzero{t\Downarrow}~x`
% \dom\lambda~t~\telzero{t\Downarrow}~x'`
% \cod\lambda~t~\telzero{t\Downarrow}~x';
% \lambda~t~x`
% \dom\lambda~t~\telsuc{\telzero{t\Downarrow}}{f}{f'}~\alpha`
% \cod\lambda~t~\telsuc{\telzero{t\Downarrow}}{f}{f'}~\alpha`
% \lambda~t~x']
% \morphism(750,300)/=>/<-200,-150>[`;\lambda_\alpha]
% \efig
% \end{equation}
% %
% Formally, we have the following definitions which generate all
% $\quad\lambda ~t~x\quad$'s for all telescopes $t$ and objects $x$ in
% them.
% %
% %
% \begin{enumerate}
% \item $\Obj$ receives a new constructor:
% \[
% \AxiomC{$t : \Tel~\homcat{\mathcal{C}}{a}{b}~n$}
% \AxiomC{$f : \Obj~(t \Downarrow)$}
% \BinaryInfC{$\lambda~t~f : \Obj~
%   (\lambda\Tel~\telsuc{t}{f}{f}\Downarrow)$}
% \DisplayProof
% \]
% %
% %
% \item $\lambda\Tel$ is a function on telescopes:
% \[
% \AxiomC{$t : \Tel~\homcat{\mathcal{C}}{a}{b}~n$}
% \UnaryInfC{$\lambda\Tel~t : \Tel~\homcat{\mathcal{C}}{a}{b}~n$}
% \DisplayProof
% \]
% \begin{equation}\label{eq:lambda-tel-def}
%   \begin{array}{l}
%  \lambda\Tel~\telzero{C} ~= ~ \telzero{C}\\   
%  \lambda\Tel~(\telsuc{t}{f}{f'})~=~\telsuc{(\lambda\Tel~t)}{\dom\lambda~t~\telzero{t\Downarrow}~f}{\cod\lambda~t~\telzero{t\Downarrow}~f'}
%   \end{array}
% \end{equation}
% %
% %
% \item 
% Functions $\dom\lambda$ and $\cod\lambda$ used above have the
% following types:
% \[
% \AxiomC{$t : \Tel~\homcat{\mathcal{C}}{a}{b}~n$}
% \AxiomC{$u : \Tel~(t \Downarrow)~m$}
% \AxiomC{$x : \Obj~u\Downarrow$}
% \TrinaryInfC{$\dom\lambda~t~u~x : \Obj~(\homcat{C}{a}{b} \dblplus
%   ~ \lambda\Tel~t~ \dblplus~\dom\lambda\Tel~t~u)$}
% \DisplayProof
% \]
% \[
% \AxiomC{$t : \Tel~\homcat{\mathcal{C}}{a}{b}~n$}
% \AxiomC{$u : \Tel~(t \Downarrow)~m$}
% \AxiomC{$x : \Obj~u\Downarrow$}
% \TrinaryInfC{$\cod\lambda~t~u~x : \Obj~(\homcat{C}{a}{b} \dblplus
%   ~ \lambda\Tel~t~ \dblplus~\cod\lambda\Tel~t~u)$}
% \DisplayProof
% \]
% %
% We omit the definitions, which in full detail are a bit technical. Just
% note that the base case, $n= 0$ is defined as above for
% $\dom\lambda_0$ and $\cod\lambda_0$ respectively, and in the induction
% step $\dom\lambda~\{n = k+1\}$ is defined in terms $\cod\lambda~\{n =
% k\}$ and $\lambda~\{n = k\}$, and similarly $\cod\lambda~\{n=k+1\}$ is defined
% in terms of $\dom\lambda~\{n=k\}$ and $\lambda~\{n=k\}$ of the smaller dimension.

% \end{enumerate}

\subsection{Right units, associativity and interchange}
Similarly to $\lambda$'s we define the remaining coherence cells,
i.e. $\rho$'s to witness \emph{right units}, $\alpha$'s to witness
\emph{associativity} of composition and $\chi$'s to witness
interchange. These are defined analogically to $\lambda$'s. Below we present
only the top-level constructor forms. 
\[
\AxiomC{$t : \Tel~\homcat{\mathcal{C}}{a}{b}~n$}
\AxiomC{$f : \Obj~(t \Downarrow)$}
\BinaryInfC{$\rho~t~f : \Obj~
  (\rho\Tel~\telsuc{t}{f}{f})\Downarrow$}
\DisplayProof
\]
\smallskip
\[
\AxiomC{$t : \Tel~\homcat{\mathcal{C}}{a}{b}~n$}
\noLine
\UnaryInfC{$f : \Obj~t \Downarrow$}
\AxiomC{$u : \Tel~\homcat{\mathcal{C}}{b}{c}~n$}
\noLine
\UnaryInfC{$g : \Obj~u \Downarrow$}
\AxiomC{$v : \Tel~\homcat{\mathcal{C}}{c}{d}~n$}
\noLine
\UnaryInfC{$h : \Obj~v \Downarrow$}
\TrinaryInfC{$\alpha~t~u~v~f~g~h : \Obj~(\alpha\Tel~\telsuc{t}{f}{f}~\telsuc{u}{g}{g}~\telsuc{v}{h}{h})\Downarrow$}
\DisplayProof
\]
\smallskip
\[
\AxiomC{$u_1 : \Tel~\homcat{\mathcal{C}}{a}{b}~n$}
\AxiomC{$u_2 : \Tel~\homcat{\mathcal{C}}{b}{c}~n$}
\noLine
\BinaryInfC{$t_{11} : \Tel~\telsuc{u_1}{a_1}{b_1}\Downarrow ~ m \qquad
t_{12} : \Tel~\telsuc{u_1}{b_1}{c_1}\Downarrow ~ m$}
\noLine
\UnaryInfC{$t_{21} : \Tel~\telsuc{u_2}{a_2}{b_2}\Downarrow ~ m \qquad
  t_{22} : \Tel~\telsuc{u_2}{b_2}{c_2}\Downarrow ~ m$}
\noLine
\UnaryInfC{$\alpha_{11} : \Obj~t_{11}\Downarrow\qquad 
\alpha_{12} : \Obj~t_{12}\Downarrow
\qquad
\alpha_{21} : \Obj~t_{21}\Downarrow
\quad
\alpha_{22} : \Obj~t_{22}\Downarrow$}
\UnaryInfC{$\chi~\alpha_{11}~\alpha_{12}~\alpha_{21}~\alpha_{22}: \Obj~(\chi\Tel~t_{11}~t_{12}~t_{21}~t_{22})\Downarrow$}
\DisplayProof
\]

We also include each of $\lambda$, $\rho$, $\alpha$ and $\chi$ in
their reversed forms, denoted $\lambda^-$, $\rho^-$, $\alpha^-$ and
$\chi^-$ respectively, as each of them should be equivalences. At this
point, the only thing we have done is postulating two cells going in
the opposite directions. The following section introduces the
necessary syntax to make them true \emph{weak equivalences}.



%%% Local Variables: 
%%% mode: latex
%%% TeX-master: "weakomega2"
%%% End: 
