\section{Laws}
\label{sec:laws}

\begin{quote}
  \begin{itemize}
  \item telescope morphisms
  \item inverse
  \item lambda,rho,alpha,interchange,inverse-laws

  \end{itemize}
\end{quote}

%
In a strict $\omega$-category composition and
identities -- structure -- are accompanied by axioms expressing their fundamental
properties. Namely, composition should be \emph{associative}, and it
should satisfy the so-called \emph{interchange law}; identities should
be the \emph{units} of composition. The fact that the axioms are
equations and one can therefore replace equals for equals in
expressions has the pleasant consequence the
complexity of axioms doesn't increase with dimension. Indeed, the whole theory
for strict $\omega$-categories can be generalised without much difficulty
to categories enriched in an arbitrary monoidal
category \cite{kelly:1982}. However, once equations are cateogorified,
i.e. when equations are replaced by data -- \emph{coherence cells} -- their
complexity rises steeply with dimension. 

For example, the 1-categorical left-unity law: 
\begin{equation}\label{eq:lambda-f-strictly}
\id_b \circ f \quad = \quad f \enspace,
\end{equation}
 for all $f : a \longrightarrow b$, is replaced in a weak
$\omega$-category by a pair of 2-cells $\lambda_f: \id_b \circ f
\Longrightarrow f $ and $\lambda^{-1}_f : f \Longrightarrow \id_b \circ
f$. A similar law should hold for $\circ$ and higher cells. I.e. it
should also hold -- strictly -- that for any $\alpha: f
\Longrightarrow f'$: 
\begin{equation}\label{eq:lambda-alpha-strictly}
\id^2_b \circ \alpha \quad = \quad \alpha\enspace,
\end{equation}
where
$\id^2_b = \id_{\id_b}$. Note that \eqref{eq:lambda-alpha-strictly}
makes sense because \eqref{eq:lambda-f-strictly} holds. In the
weak case, it is not the case that the boundary of $\id^2\circ\alpha$
is the same as the boundary of $\alpha$ and it is simply not possible
to categorify \eqref{eq:lambda-alpha-strictly} by introducing a pair
of 3-cells. However, we can use $\lambda_f$ and $\lambda^{-1}_f$
to coerce $\id_b \circ f$ to $f$ and $\id_b \circ f'$ to $f'$ making
the left and right-hand sides agree. The following figure illustrates
this:
\begin{equation}\label{eq:lambda-alpha-morphism}
\lambda_\alpha \; : \;
\bfig
\morphism/{@{>}@/^1em/}/[a`b;f]
\morphism(500,0)/{@{>}@/^1em/}/[b`b;\id\,f]
\morphism/{@{>}@/^3em/}/<1000,0>[a`b;f]
\morphism(500,250)|r|/=>/<0,-100>[`;\lambda^{-1}_f]
\morphism|b|/{@{>}@/_1em/}/[a`b;f']
\morphism(500,0)|b|/{@{>}@/_1em/}/[b`b;\id\,b]
\morphism(250,50)/=>/<0,-100>[`;\alpha]
\morphism(750,50)/=>/<0,-100>[`;\id^2 b]
\morphism|b|/{@{>}@/_3em/}/<1000,0>[a`b;f']
\morphism(500,-100)|r|/=>/<0,-100>[`;\lambda_{f'}]
\efig
\quad\Rrightarrow\quad
\bfig\
\morphism/{@{>}@/^2em/}/<500,0>[a`b;f]
\morphism(250,50)/=>/<0,-100>[`;\alpha]
\morphism|b|/{@{>}@/_2em/}/<500,0>[a`b;f']
\efig
\end{equation}
% 
The reader is invited to try to write down the fourth iteration,
i.e. the domain and codomain of $\lambda_\gamma$ for $\gamma : \alpha
\Rrightarrow \alpha' : f \Longrightarrow 'f : a \longrightarrow
a'$. Note that each higher pair of $\lambda$'s can be seen as
expressing the naturality of the preceding lower lambda. 

Similarly one must introduce $\rho$'s to witness the right unit law,
$\chi$'s to witness interchange, and $\alpha$'s to witness
associativity. Note that we have chosen the example of $\lambda$'s
because of its relative simplicity. 
%
Moreover, all such coherence cells must satisfy a coherence property
basically saying that any pair of $n$-cells from $d$ to $d'$ involving
only coherence cells\footnote{identity cells can be seen as witnessing
  reflexivity of equality} must have a mediating $n+1$-cell connecting
$d$ and $d'$. Intuitively, as coherence cells are witnesses of
equality proofs the higher coherence cells are witnessing uniqueness of
equality proofs.




This combinatorial complexity of coherence cells has been the major
obstacle in the development of weak $\omega$-categories. This has led
to the development of many diverse approaches to weak
$\omega$-categories \cite{}. However the development of Type Theory
has made it possible to express all coherence cells in a closed
form. In this section we start to describe how in Type Theory all
coherence cells can be generated by induction on their dimension.

\subsection{Formalising left units}
\label{sec:lambdas}

In \eqref{eq:lambda-alpha-morphism} we made the boundaries of the left- and right-hand
sides match by applying the function:
\[
\Phi \quad \equiv \quad (l,l') \quad \mapsto \quad x\quad
\mapsto \quad l \cdot x \cdot l' \] 
%
to $(\lambda^{-1}_f, \lambda_{f'})$ and $\id^k\,b \circ \alpha )$, where
$k$ is the depth of the telescope $\alpha$ lies in. The 3-cells
$\lambda_\alpha$ and $\lambda^{-1}_\alpha$ are then introduced as
%
\[\begin{array}{ll}\lambda_\alpha & : \quad
  \Obj~(\homcat{\homcat{\homcat{\bullet}{a}{b}}{f}{f'}}{\Phi\,(\lambda^{-1}_f,\lambda_{f'}) \,
    (\id^2\,b \circ \alpha) }{\alpha})\\
\lambda^{-1}_\alpha & : \quad \Obj~(\homcat{\homcat{\homcat{\bullet}{a}{b}}{f}{f'}}{\alpha}{\Phi\,(
 \lambda^{-1}_f,\lambda_{f'})\,(\id^2\,b\circ \alpha)})
\end{array}
\enspace.\] 
%
In a diagram:
\[
\bfig
\square/<-`>`>`>/<750,500>[ \id\,b \circ f ` f  ` \id\,b \circ
f' ` f';\lambda^{-1}_f`(\id^2\,b)\,\circ\,\alpha`\alpha`\lambda_{f'}]
\morphism(300,180)/@3{->}/<200,0>[`;\lambda_\alpha]
\morphism(300,320)/@3{<-}/<200,0>[`;\lambda^{-1}_\alpha]
\efig
\]
%
These arrows should be natural in 3-cells $\gamma : \alpha \Rrightarrow
\alpha'$ in the following sense: 
\[
\bfig
\square/<-`>`>`>/<1000,750>[ \lambda_{f'}\ast (\id^2\,b \circ \alpha)
\ast \lambda^{-1}_f ` \alpha  ` \lambda_{f'}\ast(\id^2\,b \circ
\alpha')\ast \lambda^{-1}_f ` \alpha';\lambda^{-1}_\alpha` \id\,{\lambda_{f'}}\ast((\id^3\,b)\,\circ\,\gamma)\ast \id\,{\lambda^{-1}_f}`\gamma`\lambda_{\alpha'}]
\morphism(350,300)/@3{->}/<200,0>[`;\lambda_\gamma]
\morphism(350,440)/@3{<-}/<200,0>[`;\lambda^{-1}_\gamma]
\efig
\]
%
Now note that going top-left-bottom  around the square one gets
%
\[\Phi\,(\lambda^{-1}_\alpha, \lambda_{\alpha'})\,
(\Phi\,(\lambda^{-1}_f, \lambda_{f'}) \, \gamma))\enspace.\]
%
This looks to be a good basis for the recursion generating all higher
$\lambda$'s. A similar pattern occurs in the definition of the other
coherence cells.


\subsection{Formalising all coherence cells}\label{sec:formalising-coherence}
To summarise and generalise the previous section, we want to introduce
for each $\alpha$ in a telescope $t$ of depth $n$ and each $\beta$ in
a telescope $u$ of depth $n$ a cell $\Phi\,m \,\alpha \longrightarrow
\beta$ where $m$ is the data necessary to define a function
$\Obj~(\cat{t}) \rightarrow \Obj~(\cat{u})$. We will call such an $m$
a \emph{telescope morphism form $t$ to $u$}; formally $ m : t
\rightrightarrows u$. Then $\Phi$ has type $t \rightrightarrows u
\rightarrow \Obj (\cat t) \rightarrow \Obj(\cat u)$. Formally, we
define telescope morphisms as follows; in mutual recursion to their
application to telescopes and objects in telescopes:
\[
\AxiomC{$t,u : \Tel\, C\, n $}
\UnaryInfC{$t \rightrightarrows u : \Set$}
\DisplayProof
\qquad
\AxiomC{$\phantom{\bullet : \telzero \rightrightarrows \telzero}$}
\UnaryInfC{$\bullet : \telzero \rightrightarrows \telzero$}
\DisplayProof
\]
\[ 
\AxiomC{$m : t \rightrightarrows u \quad \alpha :
  \Obj\,(\telsuc{\cat{u}}{a'}{\appobj{m}{a}}) \quad
  \Obj\,(\telsuc{\cat{u}}{\appobj{m}{b}}{b'})$}
\UnaryInfC{$\telsuc{m}{\alpha}{\beta} : \telsuc{t}{a}{b}
    \rightrightarrows \telsuc{u}{a'}{b'}$}
\DisplayProof
\]
where
\[\AxiomC{$m : t \rightrightarrows u \quad t' : \Tel\, (\cat{t})\,n$}
\UnaryInfC{$\apptel{m}{t'} : \Tel\,(\cat{u})\,n$}
\DisplayProof\]
\[\AxiomC{$m : t \rightrightarrows u \quad a : \Obj\,(\cat{t} \dblplus {'t})$}
\UnaryInfC{$\appobj{m}{a} : \Obj\,(\cat{u}\dblplus \apptel{m}{t'})$}
\DisplayProof\]
\[ 
\AxiomC{$\apptel{\telzero}{t}\;=\;t$}
\DisplayProof
\quad
\AxiomC{$\apptel{\telsuc{m'}{\alpha}{\beta}}{t} \;=\; \telsuc{(\apptel{m'}{t})}{\appobj{m}{\alpha}}{\appobj{m}{\beta}})$}
\DisplayProof
\]
%
To define $\appobj{}{}$ we need the following auxiliary function,
among others, which extends a telescope relative to a hom-category on
the left.
\[
\AxiomC{$t : \Tel~(\homcat{C}{a}{b})~(n+1)$}
\UnaryInfC{$\suctel{a}{b}{t} : \Tel~C ~ n$}
\DisplayProof
\]
\quad where
\[
\suctel{a}{b}{\telzero} = \telsuc{\telzero}{a}{b}\quad
\suctel{a}{b}{(\telsuc{t}{c}{d})} = \telsuc{(\suctel{a}{b}{t})}{c}{d}
\]
Note that here $c$ and $d$ don't actually fit into the telescope
$\suctel{a}{b}{t}$ because its definitionally different from
$t$. However, it's straightforward to prove by induction that 
%
\begin{equation}\label{eq:lem-subtel} \cat{t} \quad \equiv \quad \cat{\suctel{a}{b}{t}}\enspace,
\end{equation}
% 
and use the proof to make $c$ and $d$ fit. 
However, in the interest of avoiding syntactical clutter we usually just silently assume
that $c$ and $d$ fit or can be substituted to fit the context.

We are now in a position do define $\appobj{}{}$ as follows:
The base case is trivial:
\[
\appobj{\telzero}{x} \; = \; x
\]
The hom-case follows the pattern outlined in Section \ref{sec:lambdas}. 
\[
\AxiomC{$
  \begin{array}{c}
\homcat{m'}{\alpha}{\beta} : \homcat{t}{a}{b} \rightrightarrows \homcat{u}{a'}{b'}\quad t' :
\Tel~(\cat{\homcat{t}{a}{b}}) ~ n\\
x : \Obj~{(\cat{t'})}
\end{array}
$}
\UnaryInfC{$\appobj{\homcat{m'}{\alpha}{\beta}}{x}\;=\;\id^n\beta
    \,\circ\, (\appobj{m'}{x}) \,\circ \, \id^n\alpha$}
\DisplayProof
\]
%
In summary, $\appobj{m}{x}$ is defined by induction on $m$ where in
each step the length of $m$ decreases by one and the depth of $x$, the
length of the telescope $x$ lives is, increases by one. To make the
levels match the category $x$ lives in has to be \emph{whiskered} by
the morphisms $\alpha$, $\beta$ for $m =
\telsuc{m'}{\alpha}{\beta}$. When $m = \telzero$, the recursion
stops. The careful reader will have noticed that the expression
$\appobj{m'}{x}$ above is not well typed as $x$ lives in
$\conctel{\telsuc{t}{a}{b}}{t'}$ and we need an object in
$\conctel{t}{\suctel{a}{b}{t'}}$. But this is easily fixed by
substituting using \eqref{eq:lem-subtel}. Other similar inaccuracies
are dealt with similarly. 

Here is a picture for $m =
\telsuc{\telsuc{\telzero}{\varphi}{\gamma}}{\alpha}{\beta}$, $t' = \telzero$,
$t = \telsuc{\telsuc{\telzero}{a}{b}}{f}{g}$, $u =
\telsuc{\telsuc{\telzero}{a'}{b'}}{f'}{g'}$:
\[\bfig
\node a'(0,0)[a']
\node b'(2000,0)[b']
\node a(666,0)[\appobj{m}{a}]
\node b(1372,0)[\appobj{m}{b}]
\arrow/{@{>}@/^4em/}/[a'`b';f']
\arrow|b|/{@{>}@/_4em/}/[a'`b';g']
\arrow|m|/{@{>}@/^1.5em/}/[a'`a;\varphi]
\arrow|m|/{@{>}@/^1.5em/}/[a`b;\appobj{m}{f}]
\arrow|m|/{@{>}@/^1.5em/}/[b`b';\gamma]
\arrow|m|/{@{>}@/_1.5em/}/[a'`a;\varphi]
\arrow|m|/{@{>}@/_1.5em/}/[a`b;\appobj{m}{g}]
\arrow|m|/{@{>}@/_1.5em/}/[b`b';\gamma]
\morphism(330,50)|r|/=>/<0,-100>[`;\id_\varphi]
\morphism(1705,50)|r|/=>/<0,-100>[`;\id_\gamma]
\morphism(999,50)|r|/=>/<0,-100>[`;\appobj{m}{x}]
\morphism(999,350)|r|/=>/<0,-100>[`;\alpha]
\morphism(999,-250)|r|/=>/<0,-100>[`;\beta]
\efig
\]

Having defined telescope morphisms, it's easy to define $\lambda$'s of all depths
relative to an arbitrary category. All that is needed is to telescope a morphism, $\overrightarrow{\lambda}$: 
\[\AxiomC{$t : \Tel~\homcat{C}{a}{b}~n $}
\UnaryInfC{$\overrightarrow{\lambda}\,t : (\idTel (\id \, b) \, n) \circ t
  \rightrightarrows t$}
\DisplayProof
\]
together with a new constructor of $\Obj$:
\[
\AxiomC{$t : \Tel~\homcat{C}{a}{b}\, n \quad f : \Obj (\cat{t}) $}
\UnaryInfC{$\lambda\,t\,f :
  \Obj\,(\homcat{(\cat{t})}{\appobj{\overrightarrow{\lambda}t}{(\id^n
      b\,\circ\, f)}}{f})$}
\DisplayProof
\]
where 
\[
\AxiomC{$\overrightarrow{\lambda}\,\bullet ~=~ \bullet
\qquad 
\overrightarrow{\lambda}\,(\telsuc{t'}{a}{b}) ~ = ~
\telsuc{(\overrightarrow{\lambda}\,t')}{\inv{(\lambda\,t\,a)}}{\lambda\,t\,b}$}\DisplayProof
\]
%
Here we could define a pair of constructors $\lambda$ and
$\inv{\lambda}$ for the two opposite directions of $\lambda$. Instead,
as are interested only in groupoids, we define a generic constructor
$\inv{}$ on cells in a homcategory:
\[
\AxiomC{$t : \Tel\,(\homcat{C}{a}{b})\,n$}
\AxiomC{$a : \Obj~t$}
\BinaryInfC{$\inv{a} : \Obj\,(\conctel{\homcat{C}{b}{a}}{\inv{t}})$}
\DisplayProof\]
where $\inv{}$ extends recursively to telescopes in the obvious way.



\subsection{Right units, associativity and interchange}
Similarly to $\lambda$'s we define the remaining coherence cells,
i.e. $\rho$'s to witness \emph{right units}, $\alpha$'s to witness
\emph{associativity} of composition and $\chi$'s to witness
interchange. These are defined analogically to $\lambda$'s. 

To this end, note that everything in the definition of $\lambda$ is
forced by the type of $\overrightarrow{\lambda}$. In general it's enough for
each of $\rho$, $\alpha$ of $\chi$ to give the type of the respective telescope
morphism. For right units we define:
\[
\AxiomC{$t : \Tel\,\homcat{C}{a}{b}\,n$}
\UnaryInfC{$\overrightarrow{\rho}\,t : t \circ (\idTel\,(\id\,a)\,n)  t \rightrightarrows t$}
\DisplayProof
\]
\[
\AxiomC{$t : \Tel\,\homcat{C}{a}{b} m\quad u :
  \Tel\,\homcat{C}{b}{c}\,n \quad v : \Tel\,\homcat{C}{c}{d}\,o$}
\UnaryInfC{$\overrightarrow{\alpha}\,t\,u\,v  : (v \circ u) \circ t
  \rightrightarrows v \circ(u \circ t)
$}
\DisplayProof
\]
%
The of $\overrightarrow{\chi}$ is in $\omega$ case a bit more
complicated so we start with an example. In the 2-categorical case,
the interchange law states that the two possible ways of composing the
following diagram are equal. 
\[
\bfig
\morphism|m|<500,0>[a`b;f']
\morphism/{@{>}@/^2em/}/<500,0>[a`b;f]
\morphism|b|/{@{>}@/_2em/}/<500,0>[a`b;f'']
\morphism(250,150)|r|/=>/<0,-100>[`;\varphi]
\morphism(250,-50)|r|/=>/<0,-100>[`;\varphi']
%
\morphism(500,0)|m|<500,0>[b`c;g']
\morphism(500,0)/{@{>}@/^2em/}/<500,0>[b`c;g]
\morphism(500,0)|b|/{@{>}@/_2em/}/<500,0>[b`c;g'']
\morphism(750,150)|r|/=>/<0,-100>[`;\gamma]
\morphism(750,-50)|r|/=>/<0,-100>[`;\gamma']
\efig
\]
Formally:
\begin{equation}\label{eq:2cat-interchange}
 (\gamma'\cdot\gamma)\ast(\varphi'\cdot\varphi) \quad = \quad
(\gamma'\ast \phi')\cdot(\gamma\ast\varphi)
\end{equation}
%
In the higher dimensional case the expression of \eqref{eq:2cat-interchange}
remains the same when $\ast$ is considered a composition in some $C$
and $\cdot$ is a composition ...

\begin{ondrej}I'm finishing here for today. This needs more brain than
  I currently have.
\end{ondrej}



\[
\AxiomC{$t : \Tel~\homcat{\mathcal{C}}{a}{b}~n$}
\AxiomC{$f : \Obj~(t \Downarrow)$}
\BinaryInfC{$\rho~t~f : \Obj~
  (\rho\Tel~\telsuc{t}{f}{f})\Downarrow$}
\DisplayProof
\]
\smallskip
\[
\AxiomC{$t : \Tel~\homcat{\mathcal{C}}{a}{b}~n$}
\noLine
\UnaryInfC{$f : \Obj~t \Downarrow$}
\AxiomC{$u : \Tel~\homcat{\mathcal{C}}{b}{c}~n$}
\noLine
\UnaryInfC{$g : \Obj~u \Downarrow$}
\AxiomC{$v : \Tel~\homcat{\mathcal{C}}{c}{d}~n$}
\noLine
\UnaryInfC{$h : \Obj~v \Downarrow$}
\TrinaryInfC{$\alpha~t~u~v~f~g~h : \Obj~(\alpha\Tel~\telsuc{t}{f}{f}~\telsuc{u}{g}{g}~\telsuc{v}{h}{h})\Downarrow$}
\DisplayProof
\]
\smallskip
\[
\AxiomC{$u_1 : \Tel~\homcat{\mathcal{C}}{a}{b}~n$}
\AxiomC{$u_2 : \Tel~\homcat{\mathcal{C}}{b}{c}~n$}
\noLine
\BinaryInfC{$t_{11} : \Tel~\telsuc{u_1}{a_1}{b_1}\Downarrow ~ m \qquad
t_{12} : \Tel~\telsuc{u_1}{b_1}{c_1}\Downarrow ~ m$}
\noLine
\UnaryInfC{$t_{21} : \Tel~\telsuc{u_2}{a_2}{b_2}\Downarrow ~ m \qquad
  t_{22} : \Tel~\telsuc{u_2}{b_2}{c_2}\Downarrow ~ m$}
\noLine
\UnaryInfC{$\alpha_{11} : \Obj~t_{11}\Downarrow\qquad 
\alpha_{12} : \Obj~t_{12}\Downarrow
\qquad
\alpha_{21} : \Obj~t_{21}\Downarrow
\quad
\alpha_{22} : \Obj~t_{22}\Downarrow$}
\UnaryInfC{$\chi~\alpha_{11}~\alpha_{12}~\alpha_{21}~\alpha_{22}: \Obj~(\chi\Tel~t_{11}~t_{12}~t_{21}~t_{22})\Downarrow$}
\DisplayProof
\]

We also include each of $\lambda$, $\rho$, $\alpha$ and $\chi$ in
their reversed forms, denoted $\lambda^-$, $\rho^-$, $\alpha^-$ and
$\chi^-$ respectively, as each of them should be equivalences. At this
point, the only thing we have done is postulating two cells going in
the opposite directions. The following section introduces the
necessary syntax to make them true \emph{weak equivalences}.




%%% Local Variables: 
%%% mode: latex
%%% TeX-master: "weakomega2"
%%% End: 
