\section{Laws}
\label{sec:laws}


%
In a strict $\omega$-category composition and
identities -- structure -- are accompanied by axioms expressing their fundamental
properties. Namely, composition should be \emph{associative}, and it
should satisfy the so-called \emph{interchange law}; identities should
be the \emph{units} of composition. The fact that the axioms are
equations and one can therefore replace equals for equals in
expressions has the pleasant consequence that the
complexity of axioms doesn't increase with dimension. Indeed, the whole theory
for strict $\omega$-categories can be generalised without much difficulty
to categories enriched in an arbitrary monoidal
category \cite{kelly:1982}. However, once 
the equational axioms are replaced by data -- \emph{coherence cells} -- their
complexity rises steeply with dimension. 

This combinatorial complexity of coherence cells has been a major
obstacle in the development of weak $\omega$-categories. This has led
to the development of many diverse approaches to weak
$\omega$-categories,
e.g. \cite{street87:simplexes,batanin98:monoidal-globular,baez:1998,
trimble:1999,penon:1999,leinster:2000,lumsdaine10:weak-o-categories}. Comprehensive
surveys and comparisons can be found in
\cite{leinster:survey,cheng:guidebook}. However the development of
Type Theory has made it possible to express all coherence cells in a
closed form. In this section we start to describe how in Type Theory
all coherence cells can be generated by induction on their depth.


For example, the 1-categorical left-unity law: 
\begin{equation}\label{eq:lambda-f-strictly}
\id_b \circ f \quad = \quad f \enspace,
\end{equation}
 for all $f : a \longrightarrow b$, is replaced in a weak
$\omega$-category by a pair of 2-cells $\lambda_f: \id_b \circ f
\Longrightarrow f $ and $\lambda^{-1}_f : f \Longrightarrow \id_b \circ
f$. A similar law should hold for $\circ$ and higher cells. I.e. it
should also hold in the strict case that for any $\alpha: f
\Longrightarrow f'$: 
\begin{equation}\label{eq:lambda-alpha-strictly}
\id^2_b \circ \alpha \quad = \quad \alpha\enspace,
\end{equation}
where $\id^2_b = \id_{\id_b}$. Note that
\eqref{eq:lambda-alpha-strictly} makes sense because
\eqref{eq:lambda-f-strictly} holds. In the weak case, it is not the
case that the boundary of $\id_b^2\circ\alpha$ is equal to the
boundary of $\alpha$ and it is simply not possible to categorify
\eqref{eq:lambda-alpha-strictly} by introducing a pair of 3-cells
between the left and right side of
\eqref{eq:lambda-alpha-strictly}. However, we can use $\lambda_f$ and
$\lambda^{-1}_f$ to coerce the boundary of the former, $\id_b \circ f$
and $\id_b \circ f'$, to the boundary of the latter, $f$ and 
$f'$, respectively. The following
figure illustrates this idea:
\begin{equation}\label{eq:lambda-alpha-morphism}
\lambda_\alpha \; : \;
\bfig
\morphism/{@{>}@/^1em/}/[a`b;f]
\morphism(500,0)/{@{>}@/^1em/}/[b`b;\id_b]
\morphism/{@{>}@/^3em/}/<1000,0>[a`b;f]
\morphism(500,250)|r|/=>/<0,-100>[`;\lambda^{-1}_f]
\morphism|b|/{@{>}@/_1em/}/[a`b;f']
\morphism(500,0)|b|/{@{>}@/_1em/}/[b`b;\id_b]
\morphism(250,50)/=>/<0,-100>[`;\alpha]
\morphism(750,50)/=>/<0,-100>[`;\id^2_b]
\morphism|b|/{@{>}@/_3em/}/<1000,0>[a`b;f']
\morphism(500,-100)|r|/=>/<0,-100>[`;\lambda_{f'}]
\efig
\quad\Rrightarrow\quad
\bfig\
\morphism/{@{>}@/^2em/}/<500,0>[a`b;f]
\morphism(250,50)/=>/<0,-100>[`;\alpha]
\morphism|b|/{@{>}@/_2em/}/<500,0>[a`b;f']
\efig
\end{equation}
% 
The reader is invited to try to write down the fourth iteration,
i.e. the domain and codomain of $\lambda_\gamma$ for $\gamma : \alpha
\Rrightarrow \alpha' : f \Longrightarrow 'f : a \longrightarrow
a'$. Note that each higher pair of $\lambda$'s can be seen as
expressing the naturality of the preceding lower lambda. 

Similarly one must introduce $\rho$'s to witness the right unit law,
$\chi$'s to witness interchange, and $\alpha$'s to witness
associativity. The example of $\lambda$
has been chosen because of its relative simplicity. 

Moreover, all such coherence cells must satisfy a coherence property
basically saying that any pair of $n$-cells from $d$ to $d'$ involving
only coherence cells and units\footnote{Identity cells can be seen as
  coherence cells witnessing reflexivity of equality.} must have a
mediating n+1-cell connecting $d$ and $d'$. Intuitively, as the
coherence cells $\lambda$, $\rho$, $\alpha$ and $\chi$ we have just
described witness axioms, the higher coherence cells witness their
closure under composition and identity.

\subsection{Formalising left units}
\label{sec:lambdas}

In \eqref{eq:lambda-alpha-morphism} we made the boundaries of the left- and right-hand
sides match by applying the function:
\[
\Phi \quad \equiv \quad (l,l') \quad \mapsto \quad x\quad
\mapsto \quad l' \cdot x \cdot l \] 
%
to $(\lambda^{-1}_f, \lambda_{f'})$ and $\id^2_b \circ \alpha$. The 3-cells
$\lambda_\alpha$ and $\lambda^{-1}_\alpha$ are then introduced as
%
\[\begin{array}{ll}\lambda_\alpha & : \quad
  \Obj~(\homcat{\homcat{\homcat{\bullet}{a}{b}}{f}{f'}}{\Phi\,(\lambda^{-1}_f,\lambda_{f'}) \,
    (\id^2_b \circ \alpha) }{\alpha})\\
\lambda^{-1}_\alpha & : \quad \Obj~(\homcat{\homcat{\homcat{\bullet}{a}{b}}{f}{f'}}{\alpha}{\Phi\,(
 \lambda^{-1}_f,\lambda_{f'})\,(\id^2_b\circ \alpha)})
\end{array}
\enspace.\] 
%
In a diagram:
\[
\bfig
\square/<-`>`>`>/<750,500>[ \id_b \circ f ` f  ` \id_b \circ
f' ` f';\lambda^{-1}_f`\id^2_b\,\circ\,\alpha`\alpha`\lambda_{f'}]
\morphism(300,180)/@3{->}/<200,0>[`;\lambda_\alpha]
\morphism(300,320)/@3{<-}/<200,0>[`;\lambda^{-1}_\alpha]
\efig
\]
%
These arrows should be natural in 3-cells $\gamma : \alpha \Rrightarrow
\alpha'$ in the following sense: 
\[
\bfig
\square/<-`>`>`>/<1000,750>[ \lambda_{f'}\ast (\id^2_b \circ \alpha)
\ast \lambda^{-1}_f ` \alpha  ` \lambda_{f'}\ast(\id^2_b \circ
\alpha')\ast \lambda^{-1}_f ` \alpha';\lambda^{-1}_\alpha` \id_{\lambda_{f'}}\ast(\id^3_b\,\circ\,\gamma)\ast \id_{\{\lambda^{-1}_f}`\gamma`\lambda_{\alpha'}]
\morphism(350,300)/@3{->}/<200,0>[`;\lambda_\gamma]
\morphism(350,440)/@3{<-}/<200,0>[`;\lambda^{-1}_\gamma]
\efig
\]
%
Note that going top-left-bottom  around the square one gets
%
\[\Phi\,(\lambda^{-1}_\alpha, \lambda_{\alpha'})\,
(\Phi\,(\lambda^{-1}_f, \lambda_{f'}) \, \gamma))\enspace.\]
%
This is the basic idea of the recursion generating all higher
$\lambda$'s. A similar pattern occurs in the definition of the other
coherence cells.


\subsection{Formalising all coherence cells}\label{sec:formalising-coherence}
To summarise and generalise, we want to introduce
for each $\alpha$ in a telescope $t$ of length $n$ and each $\beta$ in
a telescope $u$ of length $n$ a cell $\Phi\,m \,\alpha \longrightarrow
\beta$ where $m$ is the data necessary to define a function
$\Obj~(\cat{t}) \rightarrow \Obj~(\cat{u})$. We will call such an $m$
a \emph{telescope morphism form $t$ to $u$}; formally $ m : t
\rightrightarrows u$. Then $\Phi$ has type $t \rightrightarrows u
\rightarrow \Obj (\cat t) \rightarrow \Obj(\cat u)$. Formally, we
define telescope morphisms as follows; in mutual recursion to their
application to telescopes and objects in telescopes:
\[
\AxiomC{$t,u : \Tel\, C\, n $}
\UnaryInfC{$t \rightrightarrows u : \Set$}
\DisplayProof
\qquad
\AxiomC{$\phantom{\bullet : \telzero \rightrightarrows \telzero}$}
\UnaryInfC{$\bullet : \telzero \rightrightarrows \telzero$}
\DisplayProof
\]
\[ 
\AxiomC{$m : t \rightrightarrows u \quad \alpha :
  \Obj\,(\telsuc{\cat{u}}{a'}{\appobj{m}{a}}) \quad
  \Obj\,(\telsuc{\cat{u}}{\appobj{m}{b}}{b'})$}
\UnaryInfC{$\telsuc{m}{\alpha}{\beta} : \telsuc{t}{a}{b}
    \rightrightarrows \telsuc{u}{a'}{b'}$}
\DisplayProof
\]
where
\[\AxiomC{$m : t \rightrightarrows u \quad t' : \Tel\, (\cat{t})\,n$}
\UnaryInfC{$\apptel{m}{t'} : \Tel\,(\cat{u})\,n$}
\DisplayProof\]
\[\AxiomC{$m : t \rightrightarrows u \quad a : \Obj\,(\cat{t} \dblplus {'t})$}
\UnaryInfC{$\appobj{m}{a} : \Obj\,(\cat{u}\dblplus \apptel{m}{t'})$}
\DisplayProof\]
\[ 
\AxiomC{$\apptel{\telzero}{t}\;=\;t$}
\DisplayProof
\quad
\AxiomC{$\apptel{\telsuc{m'}{\alpha}{\beta}}{t} \;=\; \telsuc{(\apptel{m'}{t})}{\appobj{m}{\alpha}}{\appobj{m}{\beta}})$}
\DisplayProof
\]
%
To define $\appobj{}{}$ we need the following auxiliary function,
among others, which extends a telescope relative to a hom-category on
the left.
\[
\AxiomC{$t : \Tel~(\homcat{C}{a}{b})~n$}
\UnaryInfC{$\suctel{a}{b}{t} : \Tel~C ~ (n+1)$}
\DisplayProof
\]
\quad where
\[
\suctel{a}{b}{\telzero} = \telsuc{\telzero}{a}{b}\quad
\suctel{a}{b}{(\telsuc{t}{c}{d})} = \telsuc{(\suctel{a}{b}{t})}{c}{d}
\]
Note that here $c$ and $d$ don't actually fit into the telescope
$\suctel{a}{b}{t}$ because the latter is definitionally different from
$t$. However, it's straightforward to prove by induction that 
%
\begin{equation}\label{eq:lem-subtel} \cat{t} \quad \equiv \quad \cat{\suctel{a}{b}{t}}\enspace,
\end{equation}
% 
and use the proof to make $c$ and $d$ fit. 
However, in the interest of avoiding syntactical clutter we usually just silently assume
that $c$ and $d$ fit or can be substituted to fit the context.

We are now in the position do define $\appobj{}{}$ as follows:
The base case is trivial:
\[
\appobj{\telzero}{x} \; = \; x
\]
The hom-case follows the pattern outlined in Section \ref{sec:lambdas}. 
\[
\AxiomC{$
  \begin{array}{c}
\homcat{m'}{\alpha}{\beta} : \homcat{t}{a}{b} \rightrightarrows \homcat{u}{a'}{b'}\quad t' :
\Tel~(\cat{\homcat{t}{a}{b}}) ~ n\\
x : \Obj~{(\cat{t'})}
\end{array}
$}
\UnaryInfC{$\appobj{\homcat{m'}{\alpha}{\beta}}{x}\;=\;\id^n\beta
    \,\circ\, (\appobj{m'}{x}) \,\circ \, \id^n\alpha$}
\DisplayProof
\]
%
In summary, $\appobj{m}{x}$ is defined by induction on $m$ where in
each step the length of $m$ decreases by one and the depth of $x$
increases by one. To make the levels match the category of $x$ 
has to be \emph{whiskered} by the morphisms $\alpha$, $\beta$ for $m =
\telsuc{m'}{\alpha}{\beta}$. When $m = \telzero$, the recursion
stops. The meticulous reader will have noticed that the expression
$\appobj{m'}{x}$ above is not well typed as $x$ lives in
$\conctel{\telsuc{t}{a}{b}}{t'}$ and we need an object in
$\conctel{t}{\suctel{a}{b}{t'}}$. But this is easily fixed by
substituting using \eqref{eq:lem-subtel}. Other similar inaccuracies
are dealt with similarly.

Here is an illustration of $m =
\telsuc{\telsuc{\telzero}{\varphi}{\gamma}}{\alpha}{\beta}$, $t' = \telzero$,
$t = \telsuc{\telsuc{\telzero}{a}{b}}{f}{g}$, $u =
\telsuc{\telsuc{\telzero}{a'}{b'}}{f'}{g'}$:
\[\bfig
\node a'(0,0)[a']
\node b'(2000,0)[b']
\node a(666,0)[\appobj{m}{a}]
\node b(1372,0)[\appobj{m}{b}]
\arrow/{@{>}@/^4em/}/[a'`b';f']
\arrow|b|/{@{>}@/_4em/}/[a'`b';g']
\arrow|m|/{@{>}@/^1.5em/}/[a'`a;\varphi]
\arrow|m|/{@{>}@/^1.5em/}/[a`b;\appobj{m}{f}]
\arrow|m|/{@{>}@/^1.5em/}/[b`b';\gamma]
\arrow|m|/{@{>}@/_1.5em/}/[a'`a;\varphi]
\arrow|m|/{@{>}@/_1.5em/}/[a`b;\appobj{m}{g}]
\arrow|m|/{@{>}@/_1.5em/}/[b`b';\gamma]
\morphism(330,50)|r|/=>/<0,-100>[`;\id_\varphi]
\morphism(1705,50)|r|/=>/<0,-100>[`;\id_\gamma]
\morphism(999,50)|r|/=>/<0,-100>[`;\appobj{m}{x}]
\morphism(999,350)|r|/=>/<0,-100>[`;\alpha]
\morphism(999,-250)|r|/=>/<0,-100>[`;\beta]
\efig
\]

Having defined telescope morphisms, it's easy to define $\lambda$'s of all depths
relative to an arbitrary category. All that is needed is a telescope morphism, $\overrightarrow{\lambda}$: 
\[\AxiomC{$t : \Tel~\homcat{C}{a}{b}~n $}
\UnaryInfC{$\overrightarrow{\lambda}\,t : (\idTel (\id \, b) \, n) \circ t
  \rightrightarrows t$}
\DisplayProof
\]
together with a new constructor of $\Obj$:
\[
\AxiomC{$t : \Tel~\homcat{C}{a}{b}\, n \quad f : \Obj (\cat{t}) $}
\UnaryInfC{$\lambda\,t\,f :
  \Obj\,(\homcat{(\cat{t})}{\appobj{\overrightarrow{\lambda}t}{(\id^n
      b\,\circ\, f)}}{f})$}
\DisplayProof
\]
where 
\[
\AxiomC{$\overrightarrow{\lambda}\,\bullet ~=~ \bullet
\qquad 
\overrightarrow{\lambda}\,(\telsuc{t'}{a}{b}) ~ = ~
\telsuc{(\overrightarrow{\lambda}\,t')}{\inv{(\lambda\,t\,a)}}{\lambda\,t\,b}$}\DisplayProof
\]
%
Here we could define a pair of constructors $\lambda$ and
$\inv{\lambda}$ for the two opposite directions of $\lambda$. Instead,
as we are interested only in groupoids, we define a generic constructor
$\inv{}$ on all cells of a homcategory:
\[
\AxiomC{$t : \Tel\,(\homcat{C}{a}{b})\,n$}
\AxiomC{$a : \Obj~t$}
\BinaryInfC{$\inv{a} : \Obj\,(\conctel{\homcat{C}{b}{a}}{\inv{t}})$}
\DisplayProof\]
where $\inv{}$ extends recursively to telescopes in the obvious way.



\subsection{Right units, associativity and interchange}
Similarly to $\lambda$'s we define the remaining coherence cells,
i.e. $\rho$'s to witness \emph{right units}, $\alpha$'s to witness
\emph{associativity} of composition and $\chi$'s to witness
interchange. These are defined analogically to $\lambda$'s.

To this end, note that everything in the definition of $\lambda$ is
forced by the type of $\overrightarrow{\lambda}$. In general it is
enough to give, for
each of $\rho$, $\alpha$ of $\chi$,  the type of the telescope
morphism. Just as in the case of $\lambda$, it is in each case just a
``telecopisation'' of the ordinary case.
For right units and associativity we define:
\[
\AxiomC{$t : \Tel\,\homcat{C}{a}{b}\,n$}
\UnaryInfC{$\overrightarrow{\rho}\,t : t \circ (\idTel\,(\id\,a)\,n)  \rightrightarrows t$}
\DisplayProof
\]
\[
\AxiomC{$t : \Tel~\homcat{C}{a}{b}~m\quad u :
  \Tel~\homcat{C}{b}{c}~n \quad v : \Tel~\homcat{C}{c}{d}~o$}
\UnaryInfC{$\overrightarrow{\alpha}\,t\,u\,v  : (v \circ u) \circ t
  \rightrightarrows v \circ(u \circ t)
$}
\DisplayProof
\]
%
$\overrightarrow{\chi}$ is in the $\omega$ case a bit more
complicated. In the simple 2-categorical case,
the interchange law states that the two possible ways of composing the
following diagram are equal. 
\[
\bfig
\morphism|m|<500,0>[a`b;f']
\morphism/{@{>}@/^2em/}/<500,0>[a`b;f]
\morphism|b|/{@{>}@/_2em/}/<500,0>[a`b;f'']
\morphism(250,150)|r|/=>/<0,-100>[`;\varphi]
\morphism(250,-50)|r|/=>/<0,-100>[`;\varphi']
%
\morphism(500,0)|m|<500,0>[b`c;g']
\morphism(500,0)/{@{>}@/^2em/}/<500,0>[b`c;g]
\morphism(500,0)|b|/{@{>}@/_2em/}/<500,0>[b`c;g'']
\morphism(750,150)|r|/=>/<0,-100>[`;\gamma]
\morphism(750,-50)|r|/=>/<0,-100>[`;\gamma']
\efig
\]
Formally:
\begin{equation}\label{eq:2cat-interchange}
 (\gamma'\cdot\gamma)\ast(\varphi'\cdot\varphi) \quad = \quad
(\gamma'\ast \varphi')\cdot(\gamma\ast\varphi)
\end{equation}
%
In the $\omega$-case \eqref{eq:2cat-interchange}
remains syntactically the same but we consider each of $\varphi$,
$\varphi'$, $\gamma$ and $\gamma'$ in their telescopes with the generalised
notion of composability.  The following
picture illustrates the idea:
\[
\bfig\scalefactor{2.3}
\morphism/{@{>}@/^3.5em/}/<500,0>[a`b;]
\morphism|b|/{@{>}@/_3.5em/}/<500,0>[a`b;]
\morphism(500,0)/{@{>}@/^3.5em/}/<500,0>[b`c;]
\morphism(500,0)|b|/{@{>}@/_3.5em/}/<500,0>[b`c;]
\place(430,0)[\cdots]
\place(930,0)[\cdots]
\morphism(150,130)/{@{>}@/_1em/}/<0,-260>[`;]
\morphism(340,130)/{@{>}@/^1em/}/<0,-260>[`;]
%%
\place(60,35)[_{u_1}]
\place(50,0)[\cdots]
\place(560,35)[_{u_2}] 
\place(550,0)[\cdots]
\morphism(650,130)/{@{>}@/_1em/}/<0,-260>[`;]
\morphism(840,130)/{@{>}@/^1em/}/<0,-260>[`;]
%%
\morphism(150,95)/{@{>}@/^.5em/}/<200,0>[`;a_1]
\morphism(150,-95)|b|/{@{>}@/_.5em/}/<200,0>[`;c_1]
\morphism(150,0)|m|<200,0>[`;b_1]
\place(155,55)[\cdots]
\place(155,-55)[\cdots]
\place(345,55)[\cdots]
\place(345,-55)[\cdots]
\morphism(200,100)/{@{>}@/_.2em/}/<0,-90>[`;]
\morphism(300,100)/{@{>}@/^.2em/}/<0,-90>[`;]
\morphism(200,-10)/{@{>}@/_.2em/}/<0,-90>[`;]
\morphism(300,-10)/{@{>}@/^.2em/}/<0,-90>[`;]
\place(165,85)[{_{t_{11}}}]
\place(165,-15)[{_{t_{12}}}]
\place(665,85)[{_{t_{21}}}]
\place(665,-15)[{_{t_{22}}}]
\place(655,55)[\cdots]
\place(655,-55)[\cdots]
\place(845,55)[\cdots]
\place(845,-55)[\cdots]
\morphism(650,95)/{@{>}@/^.5em/}/<200,0>[`;a_2]
\morphism(650,-95)|b|/{@{>}@/_.5em/}/<200,0>[`;c_2]
\morphism(650,0)|m|<200,0>[`;b_2]
\morphism(700,100)/{@{>}@/_.2em/}/<0,-90>[`;]
\morphism(800,100)/{@{>}@/^.2em/}/<0,-90>[`;]
\morphism(700,-10)/{@{>}@/_.2em/}/<0,-90>[`;]
\morphism(800,-10)/{@{>}@/^.2em/}/<0,-90>[`;]
%
% \morphism(225,55)<50,0>[`;\alpha_{11}]
% \morphism(225,-55)<50,0>[`;\alpha_{12}]
% \morphism(725,55)<50,0>[`;\alpha_{21}]
% \morphism(725,-55)<50,0>[`;\alpha_{22}]
\efig
\]
%
Where $\cdots$ indicate telescopes of arbitrary depth although $u_1$
and $u_2$ have to be of the same length; and $t_{ij}, \; i,j \in
\{1,2\}$ have to be of the same length. 
In this situation, it is possible to form both the composition $(t_{22}\circ
t_{21})\circ(t_{12}\circ t_{11})$ and also $(t_{22}\circ
t_{12})\circ(t_{21}\circ t_{11})$. A telescope morphism from the
former to the latter telescope induces a coherence cell for
interchange. This is formalised as follows:
\[
\AxiomC{$u_1 : \Tel~\homcat{\mathcal{C}}{a}{b}~n$}
\AxiomC{$u_2 : \Tel~\homcat{\mathcal{C}}{b}{c}~n$}
\noLine
\BinaryInfC{$t_{11} :  \Tel~(\conctel{\homcat{C}{a}{b}}{u_1}) ~ m
  \qquad t_{12} : \Tel~(\conctel{\homcat{C}{a}{b}}{u_1}) ~ m$}
\noLine
\UnaryInfC{$t_{21} : \Tel~(\conctel{\homcat{C}{b}{c}}{u_2}) ~ m 
\qquad t_{22} : \Tel~(\conctel{\homcat{C}{b}{c}}{u_2})~ m$}
% 
\UnaryInfC{$\begin{array}{l}
    \chi\,{t_{11}}\,{t_{12}}\,{t_{21}}\,{t_{22}} \;:\;\\     
    \qquad(t_{22}\circ t_{21})\circ(t_{12}\circ t_{11}) \rightrightarrows (t_{22}\circ
    t_{12})\circ(t_{21}\circ t_{11})
  \end{array}$}
\DisplayProof
\]



%%% Local Variables: 
%%% mode: latex
%%% TeX-master: "weakomega2"
%%% End: 
