\section{Introduction}

The main motivation for the present work is the development of
Univalent Type Theory by Voevodsky and others \cite{voevodsky}. In a
nutshell, Univalent Type Theory is a variant of Martin-L\"of's Type
Theory where we give up the principle of uniqueness of identity proofs
(UIP) to make it possible to treat equivalence of structures
(e.g. isomorphism of sets) as equality. While Voevodsky's motivation
comes from Homotopy Theory, Univalent Type Theory has an intrinsic
type theoretic motivation in enabling us to treat abstract structures
as first class citizen making it possible to combine high level
reasoning and concrete applicationswithout unnecessary clutter. 

The central principle of Univalent Type Theory is the Univalence Axiom
which states that equality of types is weakly equivalent to weak
equivalence. Here weak equivalence is a notion motivated by homotopy
theoretic models of type theory which can be alternatively understood
as a refinement of the notion of isomorphism in the absence of
UIP. The Univalence axiom can be viewed as a strong extensionality
priniciple and indeed it implies functional extensionality. As with
functional extensionality, univalence doesn't easily fit within the
computational understanding of Type Theory, since it does not fit into
the common pattern of introduction and elimination rules. The first
author has suggested a solution of this problem for functional
extensionality \cite{alti:lics99}: we can justify extensionality by a
translation based on the setoid model. This approach was later refined
\cite{plpv08} to \emph{Observational Type Theory} which is the base
for the development of Epigram 2 \cite{epigram2}.

However, Observational Type Theory relies essentially on UIP and hence
is incompatible with Univalent Type Theory. To address this we need to
replace setoids with a structure able to model non-unique identity
proofs. A first step in this direction is the groupoid model
\cite{HS:groupoid} but this forces UIP on the next level, i.e. for
equality between equality proofs. 
\footnote{\cite{HarperLicata} have shown that the appropriate
  restriction of Univalence can be eliminated in this setting.}
To be able to model Type Theory
without UIP at any level we need to move to
$\omega$-groupoids. Moreover, the equalities we need to assume are in
general non-strict (i.e. they are not definitional equalities in the
snese of Type Theory) and hence we need to look at weak
$\omega$-groupoids. Indeed, as \cite{garner} and \cite{lumsdaine} have
shown: Type Theory with Identity types gives rise to a weak
$\omega$-groupoid. 

Our goal is to eliminate unrestricted univalence by formalizing a weak
$\omega$-groupoid model of Type Theory in Type Theory.  As a first
step we need to implement the notion of a weak $\omega$-groupoid in
Type Theory and this is what we do in the present paper.  An obvious
possibility would have been to implement a categorical notion of weak
$\omega$ groupoids (eg. based on globular operads) in Type
Theory. However, this forces us to implement many categorical notions
first generating an avoidable overhead. It also seems that a structure
with a more typetheoretic flavour is more accessible from a naive
point of view. Hence, we are looking for a more direct
type-theoretic formulation of weak $\omega$-groupoids. In the present
paper we attempt this by defining a weak $\omega$-groupoid to be a
globular set with additional structure where this structure is given
by interpreting a syntactic theory in the globular set.



% \begin{quote}
%   \begin{itemize}
%   \item 
%     Why, potential applications: eliminating univalence axiom
%   \item cite: Harber\& Licata, Lumsdene
%   \item ref to opetopes paper (related work) Batanin
%   \end{itemize}
% \end{quote}

%%% Local Variables: 
%%% mode: latex
%%% TeX-master: "weakomega2"
%%% End: 
