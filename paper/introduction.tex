\section{Introduction}

The main motivation for the present work is the development of
Univalent Type Theory by Voevodsky \cite{voevodsky}. In a
nutshell, Univalent Type Theory is a variant of Martin-L\"of's Type
Theory where we give up the principle of uniqueness of identity proofs
(UIP) to make it possible to treat equivalence of structures
(e.g. isomorphism of sets) as equality. While Voevodsky's motivation
comes from Homotopy Theory, Univalent Type Theory has an intrinsic
type theoretic motivation in enabling us to treat abstract structures
as first class citizens making it possible to combine high level
reasoning and concrete applications without unnecessary clutter. 

The central principle of Univalent Type Theory is the Univalence Axiom
which states that equality of types is weakly equivalent to weak
equivalence. Here weak equivalence is a notion motivated by homotopy
theoretic models of type theory which can be alternatively understood
as a refinement of the notion of isomorphism in the absence of
UIP. The Univalence axiom can be viewed as a strong extensionality
principle and indeed it implies functional extensionality. As with
functional extensionality, univalence doesn't easily fit within the
computational understanding of Type Theory, since it does not fit into
the common pattern of introduction and elimination rules. The first
author has suggested a solution of this problem for functional
extensionality \cite{altenkirch99:extensional-equality}: we can justify extensionality by a
translation based on the setoid model. This approach was later refined
\cite{altenkirch07:observational-equality} to \emph{Observational Type Theory} which is the base
for the development of Epigram 2 \cite{epigram2}.

However, Observational Type Theory relies essentially on UIP and hence
is incompatible with Univalent Type Theory. To address this we need to
replace setoids with a structure able to model non-unique identity
proofs. A first step in this direction is the groupoid model
\cite{hofmann98:the-groupoid-interpretation} but this forces UIP on the next level, i.e. for
equality between equality proofs. 
\footnote{\cite{licata:2011} have shown that the appropriate
  restriction of Univalence can be eliminated in this setting.}
To be able to model Type Theory
without UIP at any level we need to move to
$\omega$-groupoids. Moreover, the equalities we need to assume are in
general non-strict (i.e. they are not definitional equalities in the
sense of Type Theory) and hence we need to look at weak
$\omega$-groupoids. Indeed, as \cite{berg08:types-are} and \cite{lumsdaine10:weak-o-categories} have
shown: Type Theory with Identity types gives rise to a weak
$\omega$-groupoid. 

Our goal is to eliminate unrestricted univalence by formalizing a weak
$\omega$-groupoid model of Type Theory in Type Theory.  As a first
step we need to implement the notion of a weak $\omega$-groupoid in
Type Theory and this is what we do in the present paper.  An obvious
possibility would have been to implement a categorical notion of weak
$\omega$-groupoids (e.g. based on globular operads) in Type
Theory. However, this forces us to implement many categorical notions
first generating an avoidable overhead. It also seems that a structure
with a more type theoretic flavour is more manageable from type theory
and more accessible from a naive point of view. Hence, we are looking
for a more direct type theoretic formulation of weak
$\omega$-groupoids. In the present paper we attempt this by defining a
weak $\omega$-groupoid to be a globular set with additional structure
where this structure is given by interpreting a syntactic theory in
the globular set.

In the following text, we first discuss and define globular sets in
type theory in Section \ref{sec:glob}. In Section \ref{sec:syntax} we
start introducing the syntax of $\omega$-groupoids by defining the general
syntactical framework which includes variables contexts, categories,
and objects. We also define interpretation of the framework into a
globular set. Sections \ref{sec:structure}, \ref{sec:laws} and
\ref{sec:coherence} are concerned with definition of the syntax of
objects. Section \ref{sec:structure} describes the construction of all
units and objects; Section \ref{sec:laws} defines the construction of
all coherence cells witnessing the left and right unit laws,
associativity of composition and interchange. In Section
\ref{sec:coherence}, we complete the definition of coherence by
introducing all coherence cell between coherence cells. In Section
\ref{sec:conclusions}  we summarise the results, and provide a rough
comparison to other (categorical) approaches to weak
$\omega$-categories. 


% \begin{quote}
%   \begin{itemize}
%   \item 
%     Why, potential applications: eliminating univalence axiom
%   \item cite: Harber\& Licata, Lumsdene
%   \item ref to opetopes paper (related work) Batanin
%   \end{itemize}
% \end{quote}

%%% Local Variables: 
%%% mode: latex
%%% TeX-master: "weakomega2"
%%% End: 
