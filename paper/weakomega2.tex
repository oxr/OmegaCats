\documentclass[draft,a4paper]{IEEEtran}
%\documentclass[a4paper]{article}

\usepackage{latexsym, amsmath, amssymb}
\usepackage{amsthm}
\RequirePackage{mathrsfs}
\RequirePackage{stmaryrd} %boxes
\usepackage{diagxy}
\usepackage{bussproofs}

\newtheorem{definition}{Definition}
\newtheorem*{remark}{Remark}

\newcommand{\hollow}{\mathsf{hollow}}
\newcommand{\dblplus}{+\hspace{-4pt}+}
\newcommand{\conctel}[2]{#1 \dblplus \, #2}
\newcommand{\glob}[2]{(\hspace{-2pt}({#1},{#2})\hspace{-2pt})}
\newcommand{\ditto}{\mathrm{- " -}}
\newcommand{\mnote}[1]{\marginpar{\footnotesize{#1}}}
\newcommand{\cell}{\mathsf{cell}}
\newcommand{\obj}{\mathsf{obj}}
\newcommand{\Set}{\mathsf{Set}}
\newcommand{\Glob}{\mathsf{Glob}}
\newcommand{\Nat}{\mathbb{N}}
\newcommand{\Con}{\mathsf{Con}}
\newcommand{\Cat}{\mathsf{Cat}}
\newcommand{\Obj}{\mathsf{Obj}}
\newcommand{\Tel}{\mathsf{Tel}}
\newcommand{\suc}{\mathsf{suc}}
\newcommand{\dom}{\mathsf{dom}}
\newcommand{\cod}{\mathsf{cod}}
\newcommand{\sym}[1]{{#1}^{-1}}
\newcommand{\refl}[1]{\id~{#1}}
\newcommand{\trans}[2]{#1 \text{;} #2}
\newcommand{\domrm}{\mathrm{dom}}
\newcommand{\codrm}{\mathrm{cod}}
\newcommand{\appobj}[2]{#1 @ #2}
\newcommand{\apptel}[2]{#1\overrightarrow{@}#2}
%\newcommand{\meets}{\mathsf{meets}}
\newcommand{\meets}{\between}
\newcommand{\zeromeets}{\mathsf{zero}}
\newcommand{\sucmeets}[1]{\mathsf{suc}}
%\newcommand{\telzero}[1]{\langle{#1}\rangle}
%\newcommand{\telzero}[1]{\diamond}
\newcommand{\telzero}{\bullet}
\newcommand{\telsuc}[3]{#1[ {#2},{#3}]}
\newcommand{\cat}[1]{{#1}\Downarrow}
\newcommand{\homcat}[3]{{#1}[#2,#3]}
\newcommand{\suctel}[3]{[ #1 , #2 ] {#3}}
\newcommand{\budoteq}{\bullet_{\doteq}}
\newcommand{\homdoteqcat}[3]{{#1}[#2,#3]_{\doteq}}
\newcommand{\intpr}[1]{\llbracket #1 \rrbracket}
\newcommand{\id}{\mathsf{id}}
\newcommand{\idTel}[2]{\mathsf{idTel}\,{#1}\,{#2}}
\newcommand{\IdCat}[2]{\mathsf{IdCat}\,{#1}\,{#2}}
\newcommand{\ItId}{\mathsf{ItId}}
\newcommand{\depth}{\mathsf{depth}}
\newcommand{\preptele}[3]{\llbracket #1,#2\rrbracket #3}
\newenvironment{ondrej}{\begin{quote}\footnotesize \textbf{Ondrej:}}{\normalsize\end{quote}}
\newcommand{\substobj}[1]{\ulcorner #1 \urcorner}
\newcommand{\Iddd}[2]{{#1}\,\equiv\,{#2}}
\newcommand{\coh}{\mathsf{coh}}
\newcommand{\cohCat}{\mathsf{cohCat}}
\newcommand{\dblline}{}
\newcommand{\emptycon}{\varepsilon}
\newcommand{\var}{\mathsf{var}}
\newcommand{\wk}{\mathsf{wk}}
\newcommand{\inv}[1]{{#1}^{-1}}

\renewcommand{\to}{\longrightarrow}

% for glob
%\newcommand{\obj}{\mathrm{obj}}
%\newcommand{\hom}{\mathrm{hom}}
\newcommand{\objto}{\mathrm{obj}^{\rightarrow}}
\newcommand{\homto}{\mathrm{hom}^{\rightarrow}}
\newcommand{\Setoid}{\mathrm{Setoid}}
%\newcommand{\Glob}{\mathrm{Glob}}
\newcommand{\Prop}{\mathbf{Prop}}
\newcommand{\Unit}{\mathbf{1}}
\newcommand{\Idw}{\mathrm{Id}^\omega}



\begin{document}
\title{A Syntactical Approach to Weak $\omega$-Groupoids}

% author names and affiliations
% use a multiple column layout for up to three different
% affiliations
\author{\IEEEauthorblockN{Thorsten Altenkirch}
\IEEEauthorblockA{Functional Programming Laboratory\\
School of Computer Science\\
University of Nottingham, UK}
\and
\IEEEauthorblockN{Ondrej Rypacek}
\IEEEauthorblockA{Department of Informatics\\
King's College London\\
London, UK
Email:ondrej.rypacek@kcl.ac.uk}
}

\maketitle


\begin{abstract}
%\boldmath
  When moving to a Type Theory without proof irrelevance the notion of
  a setoid has to be generalized to the notion of a weak
  $\omega$-groupoid. As a first step in this direction we study the
  formalisation of weak $\omega$-groupoids in Type Theory. This is
  motivated by Voevodsky's proposal of univalent type theory which is
  incompatible with proof-irrelevance and the results by Lumsdaine and
  Garner/van de Berg showing that the standard eliminator for equality
  gives rise to a weak $\omega$-groupoid.
\end{abstract}
\IEEEpeerreviewmaketitle

\section{Introduction \textit{(txa)}}

The main motivation for the present work is the development of
Univalent Type Theory by Voevodsky and others \cite{voevodsky}. In a
nutshell, Univalent Type Theory is a variant of Martin-L\"of's Type
Theory where we give up the principle of uniqueness of identity proofs
(UIP) to make it possible to treat equivalence of structures
(e.g. isomorphism of sets) as equality. While Voevodsky's motivation
comes from Homotopy Theory, Univalent Type Theory has an intrinsic
type theoretic motivation in enabling us to treat abstract structures
as first class citizen making it possible to combine high level
reasoning and concrete application without unnecessary clutter. 

The central principle of Univalent Type Theory is the Univalence Axiom
which states that equality of types is weakly equivalent to weak
equivalence. Here weak equivalence is a notion motivated by homotopy
theoretic models of type theory which can be alternatively understoond
as a refinement of the notion of isomorphism in the absence of
UIP. The Univalence axiom can be understood as a strong extensionality
priniciple and indeed it implies functional extensionality. As with
functional extensionality, univalence doesn't easily fit within the
computational understanding of Type Theory, since they do not fit into
the common pattern of introduction and elimination rules. The first
author has suggested a solution of this problem for functional
extensionality \cite{alti:lics99}: we can justify extensionality by a
translation based on the setoid model. This approach was later refined
\cite{plpv08} to \emph{Observational Type Theory} which is the base
for the development of Epigram 2.

However, Observational Type Theory relies essentially on UIP and hence
is incompatible with Univalent Type Theory. To address this we need to
replace setoids with a structure taking account of non-unique identity
proofs. A first step in this direction is the groupoid model
\cite{HS:groupoid} but this forces UIP on the next level, i.e. for
equality between equality proofs. 
\footnote{Indeed, \cite{HarperLicata} have shown that the appropriate
  restriction of Univalence can be eliminated in this setting.}
To be able to model Type Theory
without UIP at any level we need to move to
$\omega$-groupoids. Moreover, the equalities we need to assume are in
general non-strict (i.e. they are not definitional equalities in the
snese of Type Theory) and hence we need to look at weak
$\omega$-groupoids. Indeed, as \cite{garner} and \cite{lumsdaine} have
shown: Type Theory with Identity types gives rise to a weak
$\omega$-groupoid. 

Our goal is to eliminate unrestricted univalence by formalizing a weak
$\omega$-groupoid model of Type Theory in Type Theory.  As a first
step we need to implement the notion of a weak $\omega$-groupoid in
Type Theory and this is what we do in the present paper.  An obvious
possibility would have been to implement a categorical notion of weak
$\omega$ groupoids (eg. based on globular operads) in Type
Theory. However, this requires us to implement many categorical
notions first and the resulting structure will not be very
type-theoretic in flavour. Hence, we are looking for a more direct
type-theoretic formulation of weak $\omega$-groupoids. In the present
paper we attempt this by defining a weak $\omega$-groupoid to be a
globular set with additional structure where this structure is given
by interpreting a syntactic theory in the globular set.



\begin{quote}
  \begin{itemize}
  \item 
    Why, potential applications: eliminating univalence axiom
  \item cite: Harber\& Licata, Lumsdene
  \item ref to opetopes paper (related work) Batanin
  \end{itemize}
\end{quote}

%%% Local Variables: 
%%% mode: latex
%%% TeX-master: "weakomega2"
%%% End: 


\section{Globular Sets}


In Type Theory we use the notion of a setoid to describe a set with a
specific equality. That is a $A : \Setoid$ is given by the
following data:
\begin{eqnarray*}
  \obj_A & : & \Set \\
  \hom_A & : & \obj_A \rightarrow \obj_A \rightarrow \Prop
\end{eqnarray*}
and proof objects $\id, \sym{-}, \trans{-}{-}$ witnessing that $\hom$ is an
equivalence relation. Here we write $\Prop$ to denote the class of sets
which have at most one inhabitant. This restriction is important when
showing that the category of setoids has certain structure, in
particular forms a model of Type Theory. That is setoids can model a
type theory with a proof-irrelevant equality. To model proof-relevant
equality we need to generalize the notion of a setoid so that the
$\hom$-sets are generalized setoids again. It is not enough to just
postulate the laws of an equivalence relation at each level, we also
need some laws how these proofs interact. On the first level we
require the laws of a groupoid, e.g. we want that $\trans{\id}{\alpha}$ is
equal to $\alpha$. Here \emph{is equal} means that they are related by the
equality relation of a setoid again. Clearly, the structure we are
looking for is a weak $\omega$-groupoid. It is the goal of this paper
to develop a formalisation of this structure. As a first step let's
ignore the proof objects (i.e. the data of an equivalence relation and
the goupoid laws etc).
We end up with a coinductive definition of a \emph{globular set} $G :
\Glob$ given by
\begin{eqnarray*}
  \obj_G & : & \Set \\
  \hom_G & : & \obj_G \rightarrow \obj_G \rightarrow \infty\,\Glob
\end{eqnarray*}
Here we use $\infty$ to indicate that $\Glob$ is defined
coinductively. More formally, $\Glob$ is the terminal coalgebra of
$\Sigma \obj : \Set. \obj \rightarrow \obj \rightarrow -$. Given globular sets
$A,B$ a morphism $f : \Glob(A , B)$ between them is given by 
\begin{eqnarray*}
  \objto_f & : & \obj_A \rightarrow \obj_B \\
  \homto_f & : & \Pi a,b:\obj_A. \Glob(\hom_A\,a\,b,\hom_B(\objto_f\,a,\objto_f\,b))
\end{eqnarray*}
Note that this definition exploits the coinductive character of
$\Glob$. Identity and composition can be defined easily by iterating
the set-theoretic definitions ad infinitum. As an example we can
define the terminal object in $\Unit : \Glob$ by the equations
\begin{eqnarray*}
  \obj_\Unit & = & \Unit_\Set \\
  \hom_\Unit\,x\,y & = & \Unit
\end{eqnarray*}
More interestingly, the globular set of identity proofs over a given
set $A$, $\Idw\,A : \Glob$ can be defined as follows:
\begin{eqnarray*}
\obj_{\Idw\,A} & = & A \\
\hom_{\Idw\,A}\,a\,b & = & \Idw\,(a = b)
\end{eqnarray*}
Our definition of globular sets is equivalent to the usual one as a
presheaf category over the diagram:
\[
0 \two^{s_0}_{t_0} 1 \two^{s_1}_{t_1} 2 \dots n \two^{s_n}_{t_n} (n+1) \dots  
\]
with the globular identities:
\begin{eqnarray*}
  t_{i+1} \circ s_i & = & s_{i+1} \circ t_i \\
  t_{i+1} \circ t_i & = & s_{i+1} \circ t_i
\end{eqnarray*}
In the example of $\Idw\,A$ the presheaf is given by a family
$F^A : \Nat \rightarrow \Set$:
\begin{eqnarray*}
  F^A\,0 & = & A\\
  F^A\,1 & = & \Sigma a,b : A,a = b\\
  F^A\,2 & = & \Sigma a,b : A,\Sigma \alpha,\beta: a = b, \alpha = \beta\\
  \vdots & \vdots & \vdots \\
  F^A\,(n+1) & = & \Sigma a,b:A,F^{a = b}\,n
\end{eqnarray*}
and source and target maps $s_i,t_I  F^A\,(n+1) \rightarrow F^A\,n$ satisfying the globular identities. 
\begin{eqnarray*}
s_0 (a,b,-) & = & a \\
t_0 (a,b,-) & = & b \\  
s_{n+1}\,(a,b,\alpha)  & = & (a , b, s_n\,\alpha) \\
t_{n+1}\,(a,b,\alpha)  & = & (a , b, t_n\,\alpha) \\
\end{eqnarray*}

It follows from the results
\cite{lumsdaine10:weak-o-categories,berg08:types-are} that the eliminator $J$ is sufficent
to establish that $\Idw\,A$ is a weak $\omega$ groupoid. In their work
this is done on the meta level reasoning about models of type
theory. In the present paper our goal is to show that this can be done
inside Type Theory.

%%% Local Variables:
%%% TeX-master: "weakomega2"
%%% End:

\section{Syntax}\label{sec:syntax}

% \begin{quote}
%   \begin{itemize}
%   \item generic syntax
%   \item Con,Cat,Obj,Var,Tel
%   \item Interpretation wrt globular sets
%   \item Syntax => globular set
%   \end{itemize}
% \end{quote}

Our goal is to specify the conditions under which a globular set is a
weak $\omega$-groupoid. This means we need to require the existence of
certain objects in various object sets within the structure. A natural
way would be to generalize the definition of a setoid and add these
components to the structure. However,  it is not clear how to capture
the coherence condition which basically says that any two morphisms
which just represent identities in the strict case should be
equal. Instead we will follow a different approach which can be
compared to the definition of environment models for the
$\lambda$-calculus: we shall define a syntax for weak
$\omega$-groupoids and then define a weak $\omega$-groupoid as a
globular set in which this syntax can be interpreted.

\subsection{The syntactical framework}
\label{sec:syntactical-framework}

We start by presenting a syntactical framework which is a syntax for
globular sets. This syntax could be used to identify any globular set
with structure (e.g. weak or strict $\omega$-categories), the specific
aspects of a weak $\omega$-groupoid will be introduced later by adding
additional syntax for objects and other auxiliary syntactic
components. 

Our framework consists of the following components which are defined
by mutual induction\footnote{This is an instance of an inductive-inductive
definition in Type Theory, see \cite{alti:catind2}.}:
\begin{description}
\item[Contexts] 
\[\mathsf{Con} : \mathsf{Set}\]
Contexts serve to formalize the existence 
of hypothetical objects which are specified by the category in which
they live. E.g. to formalize ordinary composition we have to assume that 
objects $a,b,c$ and objects in the hom categories $f : a \to b$ and $g
: b \to c$ exist to be able to form $g \circ f : a \to c$.
\item[Categories] 
\[
\AxiomC{$\Gamma : \mathsf{Con}$}
\UnaryInfC{$\mathsf{Cat}~\Gamma : \mathsf{Set}$}
\DisplayProof
\]
Categories are basically paths within a globular set. The set of
expressions for categories depends on the context, e.g. we need at
least to assume that there is one object in the top-level category to
be able to form any other categories.
\item[Objects] 
\[
\AxiomC{$C : \mathsf{Cat}~\Gamma$}
\UnaryInfC{$\mathsf{Obj}~C:\mathsf{Set}$}
\DisplayProof
\]
Given a category we define expressions which identify objects. 
As indicated above this is the main focus of the forthcoming
sections. However, a generic instance of objects are variables which
correspond to projections out of contexts.
\[
\AxiomC{$C : \mathsf{Cat}~\Gamma$}
\UnaryInfC{$\mathsf{Var}~C : \mathsf{Set}$}
\DisplayProof
\]
\end{description}
We now specify the constructors for the various sets (apart from
objects). We use unnamed variables ala deBruijn, hence contexts are
basically sequences of categories. However, note that this is a
dependent context since the well-formedness of a category expression
may depend on the previous context.
\[
\AxiomC{\mathstrut}
\UnaryInfC{$\varepsilon : \mathsf{Con}$}
\DisplayProof
\qquad
\AxiomC{$C : \mathsf{Cat}~\Gamma$}
\UnaryInfC{$(\Gamma , C) : \mathsf{Con}$}
\DisplayProof
\]
There are two ways to form category expressions: there is the top
level category denoted by $\bullet$ and given any two objects 
$a,b$ in a category $C$ we can form the hom category $C[a,b]$.
\[
\AxiomC{$\phantom{\Gamma}$}
\UnaryInfC{$\bullet : \mathsf{Cat}~\Gamma$}
\DisplayProof
\qquad
\AxiomC{$C : \mathsf{Cat}~\Gamma\quad a,\,b : \mathsf{Obj}~C$}
\UnaryInfC{$C[ \,a\,,\,b\,] : \mathsf{Cat}~\Gamma$}
\DisplayProof
\]
We introduce nameless variables in contexts as follows:
\[
\AxiomC{$\phantom{I}$}
\dblline
\UnaryInfC{$\mathsf{vz}:\mathsf{Var}~(\mathsf{wk}~C)$}
\DisplayProof
\qquad
\AxiomC{$v : \mathsf{Var}~C$}
\dblline
\UnaryInfC{$\mathsf{vs}~v : \mathsf{Var}~(\mathsf{wk}~C~D)$}
\DisplayProof
\]
where $\mathsf{wk}$ is weakening defined for categories by
induction on the structure in the obvious way: 
\[
\AxiomC{$C,\,D : \mathsf{Cat}~\Gamma$}
\UnaryInfC{$\mathsf{wk}~C~D : \mathsf{Cat}~(\Gamma,D)$}
\DisplayProof
\qquad
\begin{array}{l}
\mathsf{wk}~\bullet~D\,=\,\bullet\\
\mathsf{wk}~(\homcat{C}{a}{b})~D\,=\,(\mathsf{wk}~C~D)[\mathsf{wk}~a~D,\mathsf{wk}~b~D]
\end{array}
\]
from weakening for objects:
\[
\AxiomC{$x : \mathsf{Obj}~C$}
\AxiomC{$D:\Cat~\Gamma$}
\BinaryInfC{$\mathsf{wk}~x~D:\mathsf{Obj}~(\mathsf{wk}~C~D)$}
\DisplayProof
\]
which is a constructor of $\mathsf{Obj}$. 

Variables become objects by the following constructor of $\Obj$:
\[
\AxiomC{$v : \mathsf{Var}~C$}
\dblline
\UnaryInfC{$\mathsf{var}~v : \mathsf{Obj}~C$}
\DisplayProof
\]


We use the usual arrow notation for categories and objects. For
instance, $\bullet[a,b]$, $\bullet[a,b][f,g]$ and $\alpha :
\Obj~(\bullet[a,b][f,g])$ are pictured respectively as follows:
\[\bfig
\morphism/{}/<300,0>[a`b;]
\efig
\quad\qquad 
\bfig
\morphism/{@{>}@/^1em/}/[a`b;f]
\morphism|b|/{@{>}@/_1em/}/[a`b;g]
\efig
\qquad 
\bfig
\morphism/{@{>}@/^1em/}/[a`b;f]
\morphism|b|/{@{>}@/_1em/}/[a`b;g]
\morphism(250,80)|r|<0,-140>[`;\alpha]
\efig
\]
%
We also write, as usual, $x : a_n\longrightarrow b_n : \cdots
: a_0 \longrightarrow b_0$ for an 
$x : \Obj~(\bullet[a_0,b_0]\cdots[a_n,b_n])$. 


% This is the basic setup for the syntax of weak omega categories,
% obviously more constructors for $\mathsf{Obj}$ are needed which we
% will discussed in the rest of the text. 

\subsection{Interpretation}
\label{sec:interpretation}
Given a globular set we define what we mean by an interpretation of
the syntax. Once we have specified all the constructors for objects a
weak $\omega$-groupoid is given by such an interpretation. The
interpretation of the structural components given in the present
section is fixed. Again this is reminiscent of environment models.

An \emph{interpretation} in a globular set $G:\Glob$ is given by the
following data:
\begin{enumerate}
\item An assignment of sets to contexts:
\[
% %<<<<<<< HEAD
% \AxiomC{$o : \Obj~C\qquad x : \intpr{\Gamma}$}
% \UnaryInfC{$\intpr{o}~x : \mathsf{obj}~(\intpr{C}~x)$}
% %=======
\AxiomC{$\Gamma : \Con$}
\UnaryInfC{$\intpr{\Gamma} : \Set$}
%>>>>>>> 341cc994e3d493b5de0c47333fc330bf43f3a890
\DisplayProof
\]
\item An assignment of globular sets to category expressions:
\[
\AxiomC{$C : \Cat ~ \Gamma$}
\AxiomC{$\gamma : \intpr{\Gamma}$}
\BinaryInfC{$\intpr{C}~\gamma : \Glob$}
\DisplayProof
\]
\item An assignment of elements of object sets to object
  expressions and variables
\[
\AxiomC{$C : \Cat ~ \Gamma$}
\AxiomC{$A : \mathsf{Obj}~C$}
\AxiomC{$\gamma : \intpr{\Gamma}$}
\TrinaryInfC{$\intpr{A}~\gamma : \obj_{\intpr{C}~\gamma}$}
\DisplayProof
\qquad
\AxiomC{$C : \Cat ~ \Gamma$}
\AxiomC{$x : \mathsf{Var}~C$}
\AxiomC{$\gamma : \intpr{\Gamma}$}
\TrinaryInfC{$\intpr{x}~\gamma : \obj_{\intpr{C}~\gamma}$}
\DisplayProof
\]
\end{enumerate}
subject to the following conditions:
\begin{align*}
\intpr{\varepsilon}  & =  1 &\intpr{\mathsf{var}~x}~\gamma  & =  \intpr{x}~\gamma \\
\intpr{\Gamma , C} & =  \Sigma \gamma : \intpr{\Gamma},\intpr{C} ~ \gamma&\intpr{\wk~a}~(\gamma,b)  & = \intpr{a}~\gamma\\
\intpr{\bullet}~\gamma & =  G&\intpr{\mathsf{vz}}~(\gamma,a)  & =  a \\
\intpr{\homcat{C}{a}{b}}~\gamma & =  \mathsf{hom}_{\intpr{C}
  \gamma}~(\intpr{a} ~ \gamma)~(\intpr{b}~\gamma)&
\intpr{\mathsf{vs}\,x}~(\gamma,a)  & = \intpr{x}~\gamma
\end{align*}


%%% Local Variables: 
%%% mode: latex
%%% TeX-master: "weakomega-csl"
%%% End: 


\section{Structure}
\label{sec:structure}



A category, strict or weak, is a globular set with additional
structure. The difference between the strict and the weak case is weather
we adorn the structure with (equational) constraints or whether one instead
of axioms introduces more structure, which witnesses rather than postulates
the constraints; so-called \emph{coherence cells}. In this section we
introduce the syntax for the structure of composition and units
giving rise to syntax for what one could call a \emph{pre-monoidal
  globular category}, where composites and units are expressible but
unconstrained by coherence cells.

\subsection{Composition}\label{sec:composition}
%
\newcommand{\cC}{\mathcal{C}}
%
In the ordinary case, a category, $\mathcal{C}$, defines a partial
operation of composition on its arrows. Explicitly, for $a$, $b$, $c$
objects of $\mathcal{C}$, $f$ in $\homcat{\mathcal{C}}{a}{b}$, $g$  in
$\homcat{\mathcal{C}}{b}{c}$, there is a $gf$ in $\homcat{\mathcal{C}}{a}{c}$.
%
In the higher-dimensional case, $\cC(a,b)$ and $\cC(b,c)$ are not mere
sets but $\omega$-categories and composition extends from sets the
whole hom-categories. Informally: for $a$, $b$, $c$ as before, $f$,
$g$, $n$-cells of homcategories $\cC(a,b)$ and $\cC(b,c)$,
respectively, one requires an $n$-cell $g\circ f$ of $\cC(a,c)$. The
fact that both $f$ and $g$ are of the same relative depth with respect
to $\cC$ is important, as well the fact the homcategories of $f$ and $g$
meet at a common object, $b$, of $\cC$. Following are some
examples of valid compositions for increasing $n$:
\begin{align}
%\label{eq:comp1}
a \to^f b \to^g c
&\quad\mapsto\quad
a\to^{gf}c
&
%\label{eq:comp2}
\bfig
\morphism/{@{>}@/^1em/}/<350,0>[a`b;f]
\morphism|b|/{@{>}@/_1em/}/<350,0>[a`b;f']
\morphism(175,50)|r|<0,-100>[`;\alpha]
\morphism(350,0)/{@{>}@/^1em/}/<350,0>[b`c;g]
\morphism(350,0)|b|/{@{>}@/_1em/}/<350,0>[b`c;g']
\morphism(520,50)|r|<0,-100>[`;\beta]
\efig
&\quad\mapsto\quad
\bfig
\morphism/{@{>}@/^1em/}/<350,0>[a`c;gf]
\morphism|b|/{@{>}@/_1em/}/<350,0>[a`c;g'f']
\morphism(175,50)|r|<0,-100>[`;\beta\alpha]
\efig
% \label{eq:comp3}
% \bfig\scalefactor{1.2}
% \morphism/{@{>}@/^1em/}/[a`b;f]
% \morphism|b|/{@{>}@/_1em/}/[a`b;f']
% \morphism(180,50)/{@{>}@/_.1em/}/<0,-100>[`;\alpha]
% \morphism(320,50)|r|/{@{>}@/^.1em/}/<0,-100>[`;\alpha']
% \morphism(500,0)/{@{>}@/^1em/}/[b`c;g]
% \morphism(500,0)|b|/{@{>}@/_1em/}/[b`c;g']
% \morphism(680,50)/{@{>}@/_.1em/}/<0,-100>[`;\beta]
% \morphism(820,50)|r|/{@{>}@/^.1em/}/<0,-100>[`;\beta']
% \morphism(200,0)<100,0>[`;\gamma]
% \morphism(700,0)<100,0>[`;\delta]
% \efig
% &\quad\mapsto\quad
% \bfig
% \morphism/{@{>}@/^1em/}/[a`c;gf]
% \morphism|b|/{@{>}@/_1em/}/[a`c;g'f']
% \morphism(180,50)/{@{>}@/_.1em/}/<0,-100>[`;\beta\alpha]
% \morphism(320,50)|r|/{@{>}@/^.1em/}/<0,-100>[`;\beta'\hspace{-3pt}\alpha']
% \morphism(200,0)<100,0>[`;\delta\gamma]
% \efig\\
\end{align}
\begin{equation}
\label{eq:comp4}
\bfig\scalefactor{1.2}
\morphism/{@{>}@/^1.7em/}/[a`b;f]
\morphism|m|[a`b;f']
\morphism(180,110)/{@{>}@/_.1em/}/<0,-100>[`;\alpha]
\morphism(320,110)|r|/{@{>}@/^.1em/}/<0,-100>[`;\alpha']
\morphism(200,55)<100,0>[`;\gamma]
%
\morphism|b|/{@{>}@/_1.7em/}/[a`b;f'']
\morphism(180,-10)/{@{>}@/_.1em/}/<0,-100>[`;\beta]
\morphism(320,-10)|r|/{@{>}@/^.1em/}/<0,-100>[`;\beta']
\morphism(200,-55)|b|<100,0>[`;\delta]
\efig
\quad\mapsto\quad
\bfig\scalefactor{2}
\morphism/{@{>}@/^1.3em/}/[a`c;f]
\morphism|b|/{@{>}@/_1.3em/}/[a`c;f'']
\morphism(180,50)/{@{>}@/_.1em/}/<0,-100>[`;\beta\cdot\alpha]
\morphism(320,50)|r|/{@{>}@/^.1em/}/<0,-100>[`;\beta'\cdot\alpha']
\morphism(200,0)<100,0>[`;\delta\cdot\gamma]
\efig
\end{equation}
% 
We formalise this as follows.

\subsubsection{Telescopes}
Firstly, we have to define when two cells are \emph{composable}. To
this end we introduce \emph{telescopes}.
Informally, a telescope is a path from a category to one of its
indirect hom-categories. 
Formally, telescopes, $\Tel$, are defined below at the same
time as their \emph{concatenation} onto a category, $\dblplus$, which takes a
telescope to a category, and therefore allows us to put objects into a
telescope:
\[
\AxiomC{$C : \Cat~\Gamma$}
\AxiomC{$n : \Nat$}
\BinaryInfC{$\Tel~C~n : \Set$}
\DisplayProof
\quad
\AxiomC{$t : \Tel~C~n$}
\UnaryInfC{$C \dblplus\, t \,:\,
  \Cat~\Gamma$}
\DisplayProof
\]
Telescopes are like categories except that the base case is an
arbitrary category $C$ rather than $\bullet$\,:
\[\AxiomC{$\phantom{C}$}
\UnaryInfC{$\bullet : \Tel~C~0$}
\DisplayProof
\quad
\AxiomC{$t : \Tel~C~n \quad a, b : \Obj\,(\conctel{C}{t})$}
\UnaryInfC{$ \telsuc{t}{a}{b} : \Tel~C~(n+1)$}
\DisplayProof
\]
%
Here, we call $n$ the \emph{length of $t$}, and we say any $x :
\Obj (\conctel{C}{t})$ to be \emph{of depth $n$}.
%
Note that only the left associative reading of $\dblplus$ makes sense
so expressions like $C \dblplus\, t \dblplus\, u$ are unambiguous.

We say that an object $x : \Obj~(\conctel{C}{t})$ \emph{lies in (the
  telescope) $t$}. When $t$ is relative to a category $C$, $x$ is
called an \emph{object relative to $C$.} 
%
Alternatively, when the path from $C$ to its subcategory is not
important we use the following syntactical shorthand:
\[
\AxiomC{$C : \Cat~\Gamma \quad t : \Tel~C~n$}
\UnaryInfC{$\cat{t}\, : \Cat~\Gamma \qquad \cat{t}\,=\,\conctel{C}{t}$}
\DisplayProof
\]

For example, given the category 
$\bfig
\morphism(0,1000)/{@{>}@/^2em/}/<1000,0>[a`b;f]
\morphism(0,1000)|b|/{@{>}@/_2em/}/<1000,0>[a`b;g]
\morphism(350,1150)|l|/{@{>}@/_.5em/}/<0,-300>[`;\varphi]
\morphism(650,1150)|r|/{@{>}@/^.5em/}/<0,-300>[`;\gamma]
\efig
$, one has:
\[\cat{\telsuc{\telzero}{\varphi}{\gamma}}~ = ~
\conctel{\homcat{\homcat{\bullet}{a}{b}}{f}{g}}{\telsuc{\telzero}{\varphi}{\gamma}} ~ = ~ \homcat{\homcat{\homcat{\bullet}{a}{b}}{f}{g}}{\varphi}{\gamma}\enspace.\]
%




\subsubsection{Back to composition}
We use telescopes to define syntax for all compositions of an
$\omega$-category. These are defined mutually recursively with
their extensions to telescopes:
%
\[
\AxiomC{$
\alpha : \Obj (t \Downarrow)
\qquad 
\beta : \Obj (u \Downarrow)
$}
\UnaryInfC{$\beta\circ \alpha : \Obj (u \circ t \Downarrow)$}
\DisplayProof
\qquad
\AxiomC{$t : \Tel~(\homcat{C}{a}{b})~n \qquad u :
  \Tel~(\homcat{C}{b}{c})~n$}
\UnaryInfC{$u \circ  t  : \Tel~(\homcat{C}{a}{c})$}
\DisplayProof
\]
Where $\circ$ is a new constructor of $\Obj$ and $\circ$ for telescopes 
is a function defined by cases
\begin{align*}
\telzero \circ \telzero & = \telzero\\
\telsuc{u}{a'}{b'} \circ \telsuc{t}{a}{b} & = \telsuc{(u\circ t) }{a' \circ a}{b' \circ b}
\end{align*}
% 
Any $\alpha$ and $\beta$ as above are said to be \emph{composable}.
Note that for a fixed category $C$, $\circ$ always defines the
composition in $C$, called \emph{horizontal} in the 2-categorical
case, which can be applied to all composable $(n+1)$-cells of $C$, where
$n$ is the length of the telescopes $t$ and $u$. To compose cells
``vertically'', one moves to a homcategory. In
2-category theory, horizontal composition is usually denoted $\circ$ or
$\ast$ or is left out, whereas vertical composition by
$\cdot\,$. In our case, we always use $\circ$ and
the level we mean is contained in the (implicit) parameter  $C$. For
example, examples in \eqref{eq:comp1} are both horizontal
compositions where $C = \bullet$, while \eqref{eq:comp4} is a
vertical composition where $C = \homcat{\bullet}{a}{b}$. 



\subsection{Units}\label{sec:units}
Any $n$-cell has $n$ compositions and so it should have $n$ units. We
generate all from a single constructor $\id$ defined as follows:
\[
\AxiomC{$a : \Obj~C$}
\UnaryInfC{$\id~a : \Obj ~\homcat{C}{a}{a}$}
\DisplayProof
\]
%
By iteration we obtain the unit for horizontal composition of $n$-cells:
\[
\AxiomC{$a : \Obj~C$}
\AxiomC{$ n : \Nat$}
\BinaryInfC{$ \idTel{a}{n} : \Tel ~C~n \qquad \id^n a : \Obj~ (\idTel{a}{n} \Downarrow) $}
\DisplayProof
\]
Again, an iterated unit is defined at the same time as its
telescope.
\begin{align*}
\idTel{a}{0} &= \telzero &\id^0 a & = a\\
\idTel{a}{(n+1)}& = \telsuc{(\idTel{a}{n})}{\id^n a}{\id^n a}&\id^{(n+1)}a&= \id\, (\id^n a)
\end{align*}




%%% Local Variables: 
%%% mode: latex
%%% TeX-master: "weakomega2"
%%% End: 



\section{Laws}
\label{sec:laws}

\begin{quote}
  \begin{itemize}
  \item telescope morphisms

  \item inverse

  \item lambda,rho,alpha,interchange,inverse-laws

  \end{itemize}
\end{quote}

\subsection{Left-unit coherence cells}
%
In a strict $\omega$-category units are accompanied by laws making
them unital with respect to composition. In the weak setting such laws
are replaced by coherence cells, which we discuss next.

With the development of the previous two sections we can express
compositions such as:
\[
\id\,b \circ f \qquad \id^2\,b \circ \alpha\qquad \id\,f'\circ \alpha
\]
pictured respectively from left to right as
\[
\bfig
\morphism[a`b;f]
\morphism(500,0)<250,0>[b`b;\id\,f]
\efig
\qquad
\bfig
\morphism/{@{>}@/^1em/}/[a`b;f]
\morphism|b|/{@{>}@/_1em/}/[a`b;f']
\morphism(250,50)<0,-100>[`;\alpha]
\morphism(500,0)/{@{>}@/^1em/}/[b`b;\id\,b]
\morphism(500,0)|b|/{@{>}@/_1em/}/[b`b;\id\,b]
\morphism(750,50)<0,-100>[`;\id^2 b]
\efig
\]\[
\bfig\scalefactor{1.2}
\morphism/{@{>}@/^1.7em/}/[a`b;f]
\morphism|-|[a`b;f']
\morphism(250,110)<0,-80>[`;\alpha]
%
\morphism|b|/{@{>}@/_1.7em/}/[a`b;f']
\morphism(250,-20)<0,-80>[`;\id\,f']
\efig
\enspace.\]
We want to define the cells that connect these cells to the cells:
\[
\bfig
\morphism[a`b;f]
\efig
\qquad
\bfig
\morphism/{@{>}@/^1em/}/[a`b;f]
\morphism|b|/{@{>}@/_1em/}/[a`b;f']
\morphism(250,50)<0,-100>[`;\alpha]
\efig
\qquad
\bfig
\morphism/{@{>}@/^1em/}/[a`b;f]
\morphism|b|/{@{>}@/_1em/}/[a`b;f']
\morphism(250,50)<0,-100>[`;\alpha]
\efig
\]

Let us analyse the situation at hand.

Firstly, note that the right-most case is just the left-most case in
the category $\homcat{\bullet}{a}{b}$ so from now on we assume
everything is taking place in an arbitrary \emph{base category},
$\cC$.

For the left-most case, assuming $a$ and $b$ are 0-cells of
  $\cC$, we simply introduce the syntax for a 2-cell $\lambda_f$ of $\cC$:
\[\bfig
\qtriangle[a`b`b;f`f`\id\,b]
\morphism(400,400)/<=/<-100,-100>[`;\lambda_f]
\efig\]

Having defined a 2-cell $\lambda_f$ for
  every 1-cell $f$, one can define a 3-cell $\lambda_\alpha$ for every
  2-cell $\alpha : f \Rightarrow f'$ as going from the left-hand side
  to the right-hand side below:
\[\bfig
\morphism/{@{>}@/^1em/}/[a`b;f]
\morphism(500,0)/{@{>}@/^1em/}/[b`b;\id\,f]
\morphism/{@{>}@/^3em/}/<1000,0>[a`b;f]
\morphism(500,250)/=>/<0,-100>[`;\lambda_f]
\morphism|b|/{@{>}@/_1em/}/[a`b;f']
\morphism(500,0)|b|/{@{>}@/_1em/}/[b`b;\id\,b]
\morphism(250,50)/=>/<0,-100>[`;\alpha]
\morphism(750,50)/=>/<0,-100>[`;\id^2 b]
\efig
\quad\Rrightarrow\quad
\bfig
\morphism/{@{>}@/^3em/}/<1000,0>[a`b;f]
\morphism|m|/{@{>}@/^1.6em/}/<1000,0>[a`b;f']
\morphism(500,290)/=>/<0,-100>[`;\alpha]
\morphism|b|/{@{>}@/_1em/}/[a`b;f']
\morphism(500,0)|b|/{@{>}@/_1em/}/[b`b;\id\,b]
\morphism(750,90)/=>/<0,-100>[`;\lambda_{f'}]
\efig
\]
Note however that because we are not working in a strict setting
the pasting together of the left-hand side is not unique. We pick a
particular one:
\[
\lambda_\alpha ~:~ ((\id^2 b) \circ \alpha) \cdot \lambda_f \Rrightarrow \lambda_{f'}\cdot
\alpha\enspace,
\]
where $\circ$ denotes composition in $\cC$ and $\cdot$ denotes
composition in $\homcat{\cC}{a}{b}$. In $\homcat{\cC}{a}{b}$ this is
the diagonal filler in the square
\begin{equation}\label{eq:lambda-natur}
\bfig
\Square[f`(\id\,b) \circ f` f'`(\id\,b)\circ f';\lambda_f`\alpha`(\id^2
b) \circ \alpha`\lambda_{f'}]
\morphism(500,300)/=>/<-200,-150>[`;\lambda_\alpha]
\efig
\end{equation}

This diagram expresses the naturality of the assignment $f \mapsto
\lambda_f$ with respect to cells $\alpha : f \Rightarrow f'$ 
witnessed by $\lambda_\alpha$. 
%

When we put:
\[
\AxiomC{$t : \Tel~\homcat{C}{a}{b}~n$}
\AxiomC{$x : \Obj~(t\Downarrow)$}
\BinaryInfC{$\dom\lambda_0~x~=~x \qquad
  \cod\lambda_0~x~=~\id^n(\id~b)\circ x$}
\DisplayProof
\]
\[
\AxiomC{$t : \Tel~\homcat{C}{a}{b}~n$}
\AxiomC{$x : \Obj~(t\Downarrow)$}
\BinaryInfC{$\lambda_n x : \Obj~(\telsuc{t}{\dom\lambda_n\, x
  }{\cod\lambda_n\,x}\Downarrow)$}
\DisplayProof
\]
then \eqref{eq:lambda-natur} can be rewritten as follows:
\begin{equation}\label{eq:lambda-natur2}
\bfig
\Square[\dom\lambda_0\,f`\cod\lambda_0\,f`\dom\lambda_0~f'`\cod\lambda_0~f';
\lambda_0 f`\dom\lambda_0~\alpha`\cod\lambda_0~\alpha`\lambda_0 f']
\morphism(500,300)/=>/<-200,-150>[`;\lambda_1 \alpha]
\efig\enspace,
\end{equation}
where necessarily 
\[
\AxiomC{$\dom\lambda_1\,\alpha = \cod\lambda_0\,\alpha\cdot
  \lambda_0\,f$}
\DisplayProof
\quad \AxiomC{$\cod\lambda_1\,\alpha = \lambda_0\, f'\cdot
  \dom\lambda_0\,\alpha$}
\DisplayProof
\]
Here $\cdot$ denotes vertical composition of 2-cells and $\circ$ is,
as before, horizontal composition.

%
To generalise and iterate \eqref{eq:lambda-natur2} we
introduce an $n+1$ cell $\quad\lambda~t~x\quad$ for each telescope
$\quad t :\Tel~\homcat{C}{a}{b}~n\quad$ and $\quad x :
\Obj~(t\Downarrow)\quad$. At the same time we introduce functions
$\dom\lambda$ and $\cod\lambda$ taking a telescope $t$, a telescope
$u$ relative to $t$, formally $u : \Tel ~ (t \Downarrow)~ m$ and a
cell $x : \Obj (u\Downarrow)$ to a new telescope and a cell in it so
that the following makes sense:
\begin{equation}\label{eq:lambda-natur3}
\bfig
\Square|ammb|[\dom\lambda~t~\telzero{t\Downarrow}~x`
\cod\lambda~t~\telzero{t\Downarrow}~x`
\dom\lambda~t~\telzero{t\Downarrow}~x'`
\cod\lambda~t~\telzero{t\Downarrow}~x';
\lambda~t~x`
\dom\lambda~t~\telsuc{\telzero{t\Downarrow}}{f}{f'}~\alpha`
\cod\lambda~t~\telsuc{\telzero{t\Downarrow}}{f}{f'}~\alpha`
\lambda~t~x']
\morphism(750,300)/=>/<-200,-150>[`;\lambda_\alpha]
\efig
\end{equation}
%
Formally, we have the following definitions which generate all
$\quad\lambda ~t~x\quad$'s for all telescopes $t$ and objects $x$ in
them.
%
%
\begin{enumerate}
\item $\Obj$ receives a new constructor:
\[
\AxiomC{$t : \Tel~\homcat{\mathcal{C}}{a}{b}~n$}
\AxiomC{$f : \Obj~(t \Downarrow)$}
\BinaryInfC{$\lambda~t~f : \Obj~
  (\lambda\Tel~\telsuc{t}{f}{f}\Downarrow)$}
\DisplayProof
\]
%
%
\item $\lambda\Tel$ is a function on telescopes:
\[
\AxiomC{$t : \Tel~\homcat{\mathcal{C}}{a}{b}~n$}
\UnaryInfC{$\lambda\Tel~t : \Tel~\homcat{\mathcal{C}}{a}{b}~n$}
\DisplayProof
\]
\begin{equation}\label{eq:lambda-tel-def}
  \begin{array}{l}
 \lambda\Tel~\telzero{C} ~= ~ \telzero{C}\\   
 \lambda\Tel~(\telsuc{t}{f}{f'})~=~\telsuc{(\lambda\Tel~t)}{\dom\lambda~t~\telzero{t\Downarrow}~f}{\cod\lambda~t~\telzero{t\Downarrow}~f'}
  \end{array}
\end{equation}
%
%
\item 
Functions $\dom\lambda$ and $\cod\lambda$ used above have the
following types:
\[
\AxiomC{$t : \Tel~\homcat{\mathcal{C}}{a}{b}~n$}
\AxiomC{$u : \Tel~(t \Downarrow)~m$}
\AxiomC{$x : \Obj~u\Downarrow$}
\TrinaryInfC{$\dom\lambda~t~u~x : \Obj~(\homcat{C}{a}{b} \dblplus
  ~ \lambda\Tel~t~ \dblplus~\dom\lambda\Tel~t~u)$}
\DisplayProof
\]
\[
\AxiomC{$t : \Tel~\homcat{\mathcal{C}}{a}{b}~n$}
\AxiomC{$u : \Tel~(t \Downarrow)~m$}
\AxiomC{$x : \Obj~u\Downarrow$}
\TrinaryInfC{$\cod\lambda~t~u~x : \Obj~(\homcat{C}{a}{b} \dblplus
  ~ \lambda\Tel~t~ \dblplus~\cod\lambda\Tel~t~u)$}
\DisplayProof
\]
%
We omit the definitions, which in full detail are a bit technical. Just
note that the base case, $n= 0$ is defined as above for
$\dom\lambda_0$ and $\cod\lambda_0$ respectively, and in the induction
step $\dom\lambda~\{n = k+1\}$ is defined in terms $\cod\lambda~\{n =
k\}$ and $\lambda~\{n = k\}$, and similarly $\cod\lambda~\{n=k+1\}$ is defined
in terms of $\dom\lambda~\{n=k\}$ and $\lambda~\{n=k\}$ of the smaller dimension.

\end{enumerate}

\subsection{Right units, associativity and interchange}
Similarly to $\lambda$'s we define the remaining coherence cells,
i.e. $\rho$'s to witness \emph{right units}, $\alpha$'s to witness
\emph{associativity} of composition and $\chi$'s to witness
interchange. These are defined analogically to $\lambda$'s. Below we present
only the top-level constructor forms. 
\[
\AxiomC{$t : \Tel~\homcat{\mathcal{C}}{a}{b}~n$}
\AxiomC{$f : \Obj~(t \Downarrow)$}
\BinaryInfC{$\rho~t~f : \Obj~
  (\rho\Tel~\telsuc{t}{f}{f})\Downarrow$}
\DisplayProof
\]
\smallskip
\[
\AxiomC{$t : \Tel~\homcat{\mathcal{C}}{a}{b}~n$}
\noLine
\UnaryInfC{$f : \Obj~t \Downarrow$}
\AxiomC{$u : \Tel~\homcat{\mathcal{C}}{b}{c}~n$}
\noLine
\UnaryInfC{$g : \Obj~u \Downarrow$}
\AxiomC{$v : \Tel~\homcat{\mathcal{C}}{c}{d}~n$}
\noLine
\UnaryInfC{$h : \Obj~v \Downarrow$}
\TrinaryInfC{$\alpha~t~u~v~f~g~h : \Obj~(\alpha\Tel~\telsuc{t}{f}{f}~\telsuc{u}{g}{g}~\telsuc{v}{h}{h})\Downarrow$}
\DisplayProof
\]
\smallskip
\[
\AxiomC{$u_1 : \Tel~\homcat{\mathcal{C}}{a}{b}~n$}
\AxiomC{$u_2 : \Tel~\homcat{\mathcal{C}}{b}{c}~n$}
\noLine
\BinaryInfC{$t_{11} : \Tel~\telsuc{u_1}{a_1}{b_1}\Downarrow ~ m \qquad
t_{12} : \Tel~\telsuc{u_1}{b_1}{c_1}\Downarrow ~ m$}
\noLine
\UnaryInfC{$t_{21} : \Tel~\telsuc{u_2}{a_2}{b_2}\Downarrow ~ m \qquad
  t_{22} : \Tel~\telsuc{u_2}{b_2}{c_2}\Downarrow ~ m$}
\noLine
\UnaryInfC{$\alpha_{11} : \Obj~t_{11}\Downarrow\qquad 
\alpha_{12} : \Obj~t_{12}\Downarrow
\qquad
\alpha_{21} : \Obj~t_{21}\Downarrow
\quad
\alpha_{22} : \Obj~t_{22}\Downarrow$}
\UnaryInfC{$\chi~\alpha_{11}~\alpha_{12}~\alpha_{21}~\alpha_{22}: \Obj~(\chi\Tel~t_{11}~t_{12}~t_{21}~t_{22})\Downarrow$}
\DisplayProof
\]

We also include each of $\lambda$, $\rho$, $\alpha$ and $\chi$ in
their reversed forms, denoted $\lambda^-$, $\rho^-$, $\alpha^-$ and
$\chi^-$ respectively, as each of them should be equivalences. At this
point, the only thing we have done is postulating two cells going in
the opposite directions. The following section introduces the
necessary syntax to make them true \emph{weak equivalences}.



%%% Local Variables: 
%%% mode: latex
%%% TeX-master: "weakomega2"
%%% End: 



\section{Coherence}
\label{sec:coherence}

\begin{quote}
  \begin{itemize}
  \item hollow (~ closed in a sense)
  \end{itemize}
\end{quote}

\subsection{Coherence cells between coherence cells}
Finally we add all coherence cells between parallel coherence
cells and their compositions. To this end we define a predicate
$\hollow$ on objects:
\[
\AxiomC{$x :  \Obj~C$}
\UnaryInfC{$\hollow~x~:~\Set$}
\DisplayProof
\]
%
Such that all identities, $\alpha$'s, $\rho$'s, $\lambda$'s and
$\chi$'s are hollow, and all weakenings and compositions of
hollow cells are hollow. Then we introduce a constructor of $\Obj$:
\[
\AxiomC{$f ~ g : \Obj~\homcat{C}{a}{b}$}
\AxiomC{$p : \hollow~f$}
\AxiomC{$q : \hollow~g$}
\TrinaryInfC{$\mathsf{coh}~f~g~p~q :
  \Obj~\homcat{\homcat{C}{a}{b}}{f}{g}$}
\DisplayProof
\]
%
And we define $\hollow~(\mathsf{coh}~f~g~p~q) = \top$.




%%% Local Variables: 
%%% mode: latex
%%% TeX-master: "weakomega2"
%%% End: 



\section{Id$\omega$ is an $\omega$-Groupoid \textit{(txa,oxr)}}
\label{sec:idw}

\begin{quote}
  \begin{itemize}
  \item Idw is an w-Groupoid (txa,oxr)
  \end{itemize}
\end{quote}
%%% Local Variables: 
%%% mode: latex
%%% TeX-master: "weakomega2"
%%% End: 


\section{Conclusions and Further Work}
\label{sec:conclusions}

\subsection{Summary}
We have presented a novel approach to defining weak $\omega$-groupoids
which is based on ideas from Type Theory. The central idea is to
define the syntax of weak $\omega$-groupoids and then define a weak
$\omega$-groupoid as a globular set with an interpretation of the
syntax, which is where Type Theory has its greatest strength. Indeed,
we have formalized most of the material presented here in Agda
\cite{agda}.  We believe that our approach to formalization of
coherence is natural, in a way naive, since it is a natural
generalisation of the corresponding first order laws.

\subsection{Related work}
Although a direct comparison is a slippery road, so we leave it for a
separate later work. Our definition is fundamentally different in that
it's formulated in Type Theory rather than Category
Theory. Many of the more direct approaches
\cite{penon:1999,batanin98:monoidal-globular,leinster:2000} refer to
the notion of strict $\omega$-category to generate coherence
cells. However approach is not available in Type Theory because the
notion of a \emph{strict} $\omega$-category is not possible without
quotient types. 

Nevertheless, there are  
similarities of our approach to the categorical approaches, at the
intuitive level, the most important of which we list below.
\begin{itemize}
\item Penon's definition \cite{penon:1999}, similarly to us, uses
  binary rather than unbiased composition, and explicit
  identities. 
\item As we can't form quotients and compare for strict equality in
  the underlying strict monoidal set, we, similarly to Street
  \cite{street87:simplexes} generate coherence cells
  inductively. 

\item Batanin's \emph{spans} are essentially our telescopes. However
  his composition is unbiased and therefore he trees in spans, which
  are definable but of no importance to us. 
\end{itemize}
%
However, as far as we know, there is no single definition that would
integrate all these points into a single definition. 



\subsection{Further work}
The current formalisation is still
quite complicated and we hope to find ways to simplify it. One interesting
idea may be to use the syntactical approach to define opetopes based on
dependent polynomial functors (i.e. indexed containers) \cite{opetopes},
which has a very type-theoretic flavour.  

We would like to use our framework to provide a formalisation of a
variation of the results by in
\cite{lumsdaine10:weak-o-categories,berg08:types-are} by showing that
$\Idw$ is a weak $\omega$-groupoid. This is slightly different form
their results because we are working insinde Type Theory instead of on
a metalevel.

The main challenge ahead is to formalize the notion of an
$\omega$-groupoid model of Type Theory. Once this has been done we
will be able to eliminate the univalence axiom and provide a
computational interpretation of this principle.


%%% Local Variables: 
%%% mode: latex
%%% TeX-master: "weakomega2"
%%% End: 



  

\section*{Acknowledgment}


The authors would like to thank...

Peter Lumsdaine, Darin Morrison


\bibliographystyle{alpha}
%\bibliographystyle{IEEEtran}
\bibliography{IEEEabrv,weakomega2.bib}
%\bibliography{weakomega2}

\end{document}




