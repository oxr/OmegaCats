\section{Syntax}\label{sec:syntax}

% \begin{quote}
%   \begin{itemize}
%   \item generic syntax
%   \item Con,Cat,Obj,Var,Tel
%   \item Interpretation wrt globular sets
%   \item Syntax => globular set
%   \end{itemize}
% \end{quote}

Our goal is to specify the conditions under which a globular set is a
weak $\omega$-groupoid. This means we need to require the existence of
certain objects in various object sets within the structure. A natural
way would be to generalize the definition of a setoid and add these
components to the structure. However,  it is not clear how to capture
the coherence condition which basically says that any two morphisms
which just represent identities in the strict case should be
equal. Instead we will follow a different approach which can be
compared to the definition of environment models for the
$\lambda$-calculus: we shall define a syntax for weak
$\omega$-groupoids and then define a weak $\omega$-groupoid as a
globular set in which this syntax can be interpreted.

\subsection{The syntactical framework}
\label{sec:syntactical-framework}

We start by presenting a syntactical framework which is a syntax for
globular sets. This syntax could be used to identify any globular set
with structure (e.g. weak or strict $\omega$-categories), the specific
aspects of a weak $\omega$-groupoid will be introduced later by adding
additional syntax for objects and other auxiliary syntactic
components. 

Our framework consists of the following main components which are defined
by mutual induction\footnote{This is an instance of an inductive-inductive
definition in Type Theory, see \cite{alti:catind2}.}:
\begin{description}
\item[Contexts, Globular Sets \& Variables] 
\[\mathsf{Con} : \mathsf{Set} \qquad
\frac{\Gamma~:~\Con}{\GlobSet~\Gamma ~:~\Set}
\qquad
\AxiomC{$G : \GlobSet~\Gamma$}
\UnaryInfC{$\mathsf{Var}~G : \mathsf{Set}$}
\DisplayProof
\]
Contexts serve to formalize the existence of hypothetical objects
which are specified by the globular set in which they live. E.g. to
formalize ordinary composition we have to assume that objects $a,b,c$
and 1-cells $f : a \to b$ and $g : b \to c$ exist to be able to form
$g \circ f : a \to c$. 
\item[Categories] 
\[
\AxiomC{$\Gamma : \mathsf{Con}$}
\UnaryInfC{$\mathsf{Cat}~\Gamma : \mathsf{Set}$}
\DisplayProof
\]
In order to define the valid compositions of cells one needs to know
their boundaries, i.e. iterated domains and targets in the globular
case. Category expressions record this data. 
% Semantically categories
% correspond to globular sets within a given top-level globular set.
The set of expressions for categories depends on a context, e.g. we
need at least to assume that there is one object in the top-level
category to be able to form any other categories.
\item[Objects] 
\[
\AxiomC{$C : \mathsf{Cat}~\Gamma$}
\UnaryInfC{$\mathsf{Obj}~C:\mathsf{Set}$}
\DisplayProof
\]
Given a category we define expressions which identify objects lying
within the category.  As indicated above this is the main focus of the
forthcoming sections. 
\end{description}
We now specify the constructors for the various sets (apart from
objects). We use unnamed variables ala deBruijn, hence contexts are
basically sequences of categories. However, note that this is a
dependent context since the well-formedness of a category expression
may depend on the previous context. At the same time we build globular
sets from nameless variables in contexts. 
\[
\AxiomC{\mathstrut}
\UnaryInfC{$\varepsilon : \mathsf{Con}$}
\DisplayProof
\qquad
\AxiomC{$G : \GlobSet~\Gamma$}
\UnaryInfC{$(\Gamma , G) : \mathsf{Con}$}
\DisplayProof
\qquad
\frac{}{\bullet : \GlobSet~\Gamma}\qquad\frac{G :
  \GlobSet~\Gamma\quad a , b : \mathsf{Var} ~G}{G [ a , b ] : \GlobSet~\Gamma}
\]
\[
\AxiomC{$\phantom{I}$}
\dblline
\UnaryInfC{$\mathsf{vz}:\mathsf{Var}~(\mathsf{wk}~G)$}
\DisplayProof
\qquad
\AxiomC{$v : \mathsf{Var}~G$}
\dblline
\UnaryInfC{$\mathsf{vs}~v : \mathsf{Var}~(\mathsf{wk}~G~G')$}
\DisplayProof
\]
where $\mathsf{wk}$ is weakening defined for globular sets by
induction on the structure in the obvious way: 
\[
\AxiomC{$G,\,G' : \GlobSet~\Gamma$}
\UnaryInfC{$\mathsf{wk}~G~G' : \GlobSet~(\Gamma,G')$}
\DisplayProof
\qquad
\begin{array}{l}
\mathsf{wk}~\bullet~G'\,=\,\bullet\\
\mathsf{wk}~(\homcat{G}{a}{b})~G'\,=\,(\mathsf{wk}~G~G')[\mathsf{vs}~a,\mathsf{vs}~b]
\end{array}
\]
There are two ways to form category expressions: there is the top
level category denoted by $\bullet$ and given any two objects 
$a,b$ in a category $C$ we can form the hom category $C[a,b]$.
\[
\AxiomC{$\phantom{\Gamma}$}
\UnaryInfC{$\bullet : \mathsf{Cat}~\Gamma$}
\DisplayProof
\qquad
\AxiomC{$C : \mathsf{Cat}~\Gamma\quad a,\,b : \mathsf{Obj}~C$}
\UnaryInfC{$C[ \,a\,,\,b\,] : \mathsf{Cat}~\Gamma$}
\DisplayProof
\]

% from weakening for objects:
% \[
% \AxiomC{$x : \mathsf{Obj}~C$}
% \AxiomC{$D:\Cat~\Gamma$}
% \BinaryInfC{$\mathsf{wk}~x~D:\mathsf{Obj}~(\mathsf{wk}~C~D)$}
% \DisplayProof
% \]
% which is a constructor of $\mathsf{Obj}$. 

\noindent Variables become objects by the following constructor of
$\Obj$, which mutually extends to $\GlobSet$s:
\[
\AxiomC{$v : \mathsf{Var}~G$}
\dblline
\UnaryInfC{$\mathsf{var}~v : \mathsf{Obj}~(\mathsf{var}~G)$}
\DisplayProof
\qquad
\frac{G : \GlobSet~\Gamma}
{\mathsf{var}~G : \Cat~\Gamma}
\quad\text{where}\quad
\begin{array}{lcl}
\mathsf{var}~\bullet&=& \bullet\\
\mathsf{var}~\homcat{G}{a}{b}&=&\homcat{(\mathsf{var}~G)}{\mathsf{var}~a}{\mathsf{var}~b}
\end{array}
\]


We use the usual arrow notation for categories and objects. For
instance, $\bullet[a,b]$, $\bullet[a,b][f,g]$ and $\alpha :
\Obj~(\bullet[a,b][f,g])$ are pictured respectively as follows:
\[\bfig
\morphism/{}/<300,0>[a`b;]
\efig
\quad\qquad 
\bfig
\morphism/{@{>}@/^1em/}/[a`b;f]
\morphism|b|/{@{>}@/_1em/}/[a`b;g]
\efig
\qquad 
\bfig
\morphism/{@{>}@/^1em/}/[a`b;f]
\morphism|b|/{@{>}@/_1em/}/[a`b;g]
\morphism(250,80)|r|<0,-140>[`;\alpha]
\efig
\]
%
We also write, as usual, $x : a_n\longrightarrow b_n : \cdots : a_0
\longrightarrow b_0$ for an $x :
\Obj~(\bullet[a_0,b_0]\cdots[a_n,b_n])$.  Note that it is essential to
first introduce $\GlobSet$s and then $\Cat$s with an inclusion
$\mathsf{var}:\GlobSet~\Gamma \to \Cat~\Gamma$. In this way we make
sure that variables alone form a globular set, i.e. that the domain
and codomain of a variable is a variable. In particular, that it is not
possible to introduce a variable between syntactically constructed
coherence cells. In this way we can talk about the $\omega$-category
freely generated by a globular set.

% This is the basic setup for the syntax of weak omega categories,
% obviously more constructors for $\mathsf{Obj}$ are needed which we
% will discussed in the rest of the text. 

\subsection{Interpretation}
\label{sec:interpretation}
Given a globular set we define what we mean by an interpretation of
the syntax. Once we have specified all the constructors for objects a
weak $\omega$-groupoid is given by such an interpretation. The
interpretation of the structural components given in the present
section is fixed. Again this is reminiscent of environment models.

An \emph{interpretation} in a globular set $G:\Glob$ is given by the
following data:
\begin{enumerate}
\item An assignment of sets to contexts:
\[
% %<<<<<<< HEAD
% \AxiomC{$o : \Obj~C\qquad x : \intpr{\Gamma}$}
% \UnaryInfC{$\intpr{o}~x : \mathsf{obj}~(\intpr{C}~x)$}
% %=======
\AxiomC{$\Gamma : \Con$}
\UnaryInfC{$\intpr{\Gamma} : \Set$}
%>>>>>>> 341cc994e3d493b5de0c47333fc330bf43f3a890
\DisplayProof
\]
\item An assignment of globular sets to $\GlobSet$ and $\Cat$ expressions:
\[
\AxiomC{$G : \GlobSet ~ \Gamma$}
\AxiomC{$\gamma : \intpr{\Gamma}$}
\BinaryInfC{$\intpr{G}~\gamma : \Glob$}
\DisplayProof
\qquad
\AxiomC{$C : \Cat ~ \Gamma$}
\AxiomC{$\gamma : \intpr{\Gamma}$}
\BinaryInfC{$\intpr{C}~\gamma : \Glob$}
\DisplayProof
\]
\item An assignment of elements of object sets to object
  expressions and variables
\[
\AxiomC{$G : \GlobSet ~ \Gamma$}
\AxiomC{$x : \mathsf{Var}~G$}
\AxiomC{$\gamma : \intpr{\Gamma}$}
\TrinaryInfC{$\intpr{x}~\gamma : \obj_{\intpr{G}~\gamma}$}
\DisplayProof
\qquad
\AxiomC{$C : \Cat ~ \Gamma$}
\AxiomC{$A : \mathsf{Obj}~C$}
\AxiomC{$\gamma : \intpr{\Gamma}$}
\TrinaryInfC{$\intpr{A}~\gamma : \obj_{\intpr{C}~\gamma}$}
\DisplayProof
\]
\end{enumerate}
subject to the following conditions:
\[\begin{array}{lclclcl}
\intpr{\varepsilon}  & = & 1 &\quad&\intpr{\mathsf{var}~x}~\gamma  & =&  \intpr{x}~\gamma \\
\intpr{\Gamma , G} & =  &\Sigma \gamma : \intpr{\Gamma},\intpr{G} ~
\gamma&&\intpr{\mathsf{vz}}~(\gamma,a)  & = & a \\   % &&\intpr{\wk~a}~(\gamma,b)  & = &\intpr{a}~\gamma\\
\intpr{\bullet}~\gamma & = & G&&\intpr{\mathsf{vs}\,x}~(\gamma,a)  & = &\intpr{x}~\gamma\\
\intpr{\homcat{C}{a}{b}}~\gamma & = & \mathsf{hom}_{\intpr{C}
  \gamma}~(\intpr{a} ~ \gamma)~(\intpr{b}~\gamma)% && \intpr{\homcat{G}{a}{b}}~\gamma &= & \mathsf{hom}_{\intpr{G}  \gamma}~(\intpr{a} ~ \gamma)~(\intpr{b}~\gamma)
\end{array}
\enspace,\]
where the last case applies both to $\GlobSet$s and $\Cat$s.

%%% Local Variables: 
%%% mode: latex
%%% TeX-master: "weakomega-csl"
%%% End: 
