\section{Syntax \textit{(txa)}}\label{sec:syntax}

\begin{quote}
  \begin{itemize}
  \item generic syntax
  \item Con,Cat,Obj,Var,Tel
  \item Interpretation wrt globular sets
  \item Syntax => globular set
  \end{itemize}
\end{quote}

\subsection{The syntactical framework}
\label{sec:syntactical-framework}

We start introducing syntax for globular sets as follows:
\[
\mathbf{data}\;
\AxiomC{$\mathsf{Con} : \mathsf{Set}$}
\DisplayProof
\;
\mathbf{where}
\;
\AxiomC{\mathstrut}
\UnaryInfC{$\varepsilon : \mathsf{Con}$}
\DisplayProof
\;;\;
\AxiomC{$C : \mathsf{Cat}~\Gamma$}
\UnaryInfC{$(\Gamma , C) : \mathsf{Con}$}
\DisplayProof
\]
\[
\mathbf{data}\;
\AxiomC{$\Gamma : \mathsf{Con}$}
\UnaryInfC{$\mathsf{Cat}~\Gamma : \mathsf{Set}$}
\DisplayProof
\;\mathbf{where}\;
\AxiomC{$\phantom{\Gamma}$}
\UnaryInfC{$\bullet : \mathsf{Cat}~\Gamma$}
\DisplayProof
\;;\;\]\[
\AxiomC{$C : \mathsf{Cat}~\Gamma\quad a,\,b : \mathsf{Obj}~C$}
\UnaryInfC{$C[ \,a\,,\,b\,] : \mathsf{Cat}~\Gamma$}
\DisplayProof
\]
\[
\mathbf{data}\;
\AxiomC{$C : \mathsf{Cat}~\Gamma$}
\UnaryInfC{$\mathsf{Obj}~C:\mathsf{Set}$}
\DisplayProof
\;
\mathbf{where}
\;\cdots
\]
%
We introduce mutually recursively \emph{contexts}, $\mathsf{Con}$,
\emph{categories}, $\Cat$, 
and \emph{objects}, $\Obj$. Contexts are telescopes of categories;
\emph{categories} in context are built inductively from a base category,
$\bullet$, by formation of homcategories, $\homcat{C}{a}{b}$.
%
We omit the constructors for $\Obj$ for the moment. Their development
is the subject of the rest of the text.  
% We use double horizontal
% lines to introduce constructors of datatypes. 
We leave out as many assumptions as can be inferred from the
context. For example, the assumption $\Gamma : \Con$ can be inferred
in the rule for $\bullet$ introduction. In such cases when we need to
\emph{name} implicit assumptions, we use curly brackets. We also leave
out type annotations where inferable.

% In summary, expressions
% for categories are of the form 
% \[
% \bullet [~a_0~,~b_0~]\cdots[~a_k~,~b_k~]
% \]
% where each $a_i$, $b_i$ are objects in category $\bullet
% [~a_0~,~b_0~]\cdots[~a_{i-1}~,~b_{i-1}~]$. 

%
The role of contexts is to introduce irreducible objects --
variables. Using De Bruijn indexing we introduce nameless variables in
contexts as follows:
\[
\mathbf{data}\;
\AxiomC{$C : \mathsf{Cat}~\Gamma$}
\UnaryInfC{$\mathsf{Var}~C : \mathsf{Set}$}
\DisplayProof
\;\mathbf{where}\;
\AxiomC{$\phantom{I}$}
\dblline
\UnaryInfC{$\mathsf{vz}:\mathsf{Var}~(\mathsf{wk}~C)$}
\DisplayProof
\;;
\]
\[
\AxiomC{$v : \mathsf{Var}~C$}
\dblline
\UnaryInfC{$\mathsf{vs}~v : \mathsf{Var}~(\mathsf{wk}~C~D)$}
\DisplayProof
\]
where $\mathsf{wk}$ is weakening defined for categories by
induction on the structure in the obvious way: 
\[
\AxiomC{$C,\,D : \mathsf{Cat}~\Gamma$}
\UnaryInfC{$\mathsf{wk}~C~D : \mathsf{Cat}~(\Gamma,D)$}
\DisplayProof
\;\mathbf{where}\;
\]\[
\big\{
\begin{array}{l}
\mathsf{wk}~\bullet~D\,=\,\bullet\\
\mathsf{wk}~(\homcat{C}{a}{b})~D\,=\,(\mathsf{wk}~C~D)[\mathsf{wk}~a~D,\mathsf{wk}~b~D]
\end{array}
\]
% \mathsf{wk} : \{\Gamma\}()
% \rightarrow 
% \enspace,\]
%
from weakening for objects:
\[
\AxiomC{$x : \mathsf{Obj}~C$}
\AxiomC{$D:\Cat~\Gamma$}
\BinaryInfC{$\mathsf{wk}~x~D:\mathsf{Obj}~(\mathsf{wk}~C~D)$}
\DisplayProof
\enspace,
\]
which is defined as a data constructor of $\mathsf{Obj}$. 

Variables become objects by the following data constructor of $\Obj$:
\[
\AxiomC{$v : \mathsf{Var}~C$}
\dblline
\UnaryInfC{$\mathsf{var}~v : \mathsf{Obj}~C$}
\DisplayProof
\]
%
This completes the definition of globular sets with variables. To
explicitly define a globular set out of the syntactical presentation
we define globular sets coinductively as follows:

% where each
%   $G_n$ is a set in Voevodsky's sense \cite{}. \mnote{I want a proper
%     Set and the only way to say it in Type Theory is by
%     contractibility.} Formally, we define 
% $\mathsf{Glob}$ by the following \emph{coinductive}
% definition:
% % , where we use Altenkirch's \emph{delay}, $\sharp$, and
% % \emph{force}, $\flat$, on coinductive sets $S^\infty$
% %\cite{}:
\[
\AxiomC{$\phantom{\{\Gamma : \Con \}}$}
\UnaryInfC{$\mathsf{Glob}:\Set$}
\DisplayProof
\quad
\AxiomC{$ O : \Set$}
\AxiomC{$ H : ( a,\,b : \Set) \rightarrow \mathsf{Glob}$}
% \noLine
% \BinaryInfC{$C : (a ~ b : O) \rightarrow
%   \Sigma(p : a \equiv b)((q : a \equiv b) \rightarrow p \equiv q)$}
\BinaryInfC{$\glob{O}{H} : \mathsf{Glob}$}
\DisplayProof
\]
for which we introduce projections:
\[
\mathsf{obj}~\glob{x}{y} = x \qquad \mathsf{hom}~\glob{x}{y} = y 
\]
% Note that axiom $C$ in the definition of $\mathsf{glob}$ is the
% statement that propositional equality in $O$ is contractible.
\noindent
Then for each $\Gamma$ and  $C : \Cat~\Gamma$, 
\[\mathsf{glob}~C =
\glob{\Obj~C}{\lambda~x~y.\mathsf{glob}~\homcat{C}{x}{y}}\]
%
Finally we define the globular set of a context $\Gamma$ as
\begin{equation}\label{eq:glob-of-gamma}
\mathsf{glob}~\Gamma~=~\mathsf{glob}~(\bullet)
\end{equation}
%
% A \emph{globular set morphism} $\glob{O}{H} \to \glob{O'}{H'}$ is an 
% \[f : O \to O'\]
% together with a collection:
% \[h ~ x ~ y : H ~ x ~ y \to H'~(f~x)~(f~y)\enspace.\]


We will need the following definition:
\[
\AxiomC{$C : \Cat~\Gamma$}
\UnaryInfC{$\mathsf{depth}~C : \Nat$}
\DisplayProof
\;\mathbf{where}\;
\big\{{\begin{array}{l}
  \mathsf{depth}~\bullet\,=\,0\\
\mathsf{depth}~(C[a,b])~=~(\mathsf{depth}~C) + 1
\end{array}}
\quad
%\AxiomC{$C : \Cat~\Gamma$}
%\AxiomC{$a~b : \Obj~C$}
\enspace,\]
where $\Nat$ is the $\Set$ of natural numbers with constructors $0$ and $+1$.
%
% &\mathsf{depth}:(\mathsf{Cat}~\Gamma)\rightarrow \Nat\\
% &\mathsf{depth}~\bullet = 0\\
% &\mathsf{depth}~(C[a,b]) = (\mathsf{depth}~C) + 1
% \end{align*}
%
Then any $x : \Obj~C$ such that $\mathsf{depth}~C \equiv n$, where
$\equiv$ denotes propositional equality, is called an \emph{$n$-cell}.
%
We use the usual arrow notation for categories and objects. For
instance, $\bullet[a,b]$, $\bullet[a,b][f,g]$ and $\alpha :
\Obj~(\bullet[a,b][f,g])$ are pictured respectively as follows:
\[\bfig
\morphism/{}/<300,0>[a`b;]
\efig
\quad\qquad 
\bfig
\morphism/{@{>}@/^1em/}/[a`b;f]
\morphism|b|/{@{>}@/_1em/}/[a`b;g]
\efig
\qquad 
\bfig
\morphism/{@{>}@/^1em/}/[a`b;f]
\morphism|b|/{@{>}@/_1em/}/[a`b;g]
\morphism(250,80)|r|<0,-140>[`;\alpha]
\efig
\]
%
We also write, as usual, $x : a_n\longrightarrow b_n : \cdots
: a_0 \longrightarrow b_0$ for an 
$x : \Obj~(\bullet[a_0,b_0]\cdots[a_n,b_n])$. 

% This is the basic setup for the syntax of weak omega categories,
% obviously more constructors for $\mathsf{Obj}$ are needed which we
% will discussed in the rest of the text. 

\subsection{Interpretation}
\label{sec:interpretation}
A weak $\omega$-category is a globular set, $G$, with certain
constraints on existence of cells. For example, for each object $a$ in
$G$ there must be an object $\id_a$ in $(\mathsf{hom}~G~a~a)$, and for
$x$ in $(\mathsf{hom}~G~a~b)$ and $y$ in $(\mathsf{hom}~G~b~c)$ there
must be an object in $(\mathsf{hom}~G~a~c)$. Abstractly we can
formalise all such constraints by the existence of an action of the
syntax of weak $\omega$-categories on $G$. The details follow:

\begin{definition}
A \emph{weak $\omega$ category} is given by the following data:
\begin{enumerate}
\item A globular set $G : \Glob$ 
\item An action of the syntax of objects on contexts, which are
  lists of cells of $G$:
\[
\AxiomC{$o : \Obj~C\qquad x : \intpr{\Gamma}$}
\BinaryInfC{$\intpr{o}~x : \mathsf{obj}~(\intpr{C}~x)$}
\DisplayProof
\]
where the following extensions of $\intpr{\text{-}}$ to contexts,
categories and variables are used:
\[
\AxiomC{$\Gamma : \Con$}
\UnaryInfC{$\intpr{\Gamma} : \Set$}
\DisplayProof
\;\mathbf{where}\;
\begin{array}{l}
\intpr{\varepsilon} = 1\\
\intpr{\Gamma , C} = \Sigma(x : \intpr{\Gamma})(\intpr{C} ~ x)
\end{array}
\]
\[
\AxiomC{$C : \Cat ~ \Gamma$}
\AxiomC{$x : \intpr{\Gamma}$}
\BinaryInfC{$\intpr{C}~x : \Glob$}
\DisplayProof
\;\mathbf{where}\;\]\[
\big\{\begin{array}{l}
\intpr{\bullet}~x = G\\
\intpr{\homcat{C}{a}{b}}~x = \mathsf{hom}~(\intpr{C}
  x)~(\intpr{a} ~ x)~(\intpr{b}~x)
\end{array}
\]

\[\vdots\]\mnote{to do: variables as projections}
and such that
\[
\begin{array}{lcl}
\intpr{\var~v} ~=~ \intpr{v}~\intpr{\Gamma}&\quad&\intpr{\wk~x~D}  ~=~ \intpr{x}
\end{array}
\]
\end{enumerate}
\end{definition}


%%% Local Variables: 
%%% mode: latex
%%% TeX-master: "weakomega2"
%%% End: 
