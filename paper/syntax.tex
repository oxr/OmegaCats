\section{Syntax \textit{(txa)}}\label{sec:syntax}

% \begin{quote}
%   \begin{itemize}
%   \item generic syntax
%   \item Con,Cat,Obj,Var,Tel
%   \item Interpretation wrt globular sets
%   \item Syntax => globular set
%   \end{itemize}
% \end{quote}

Our goal is to specify the conditions under which a globular set is a
weak $\omega$-groupoid. This means we need to require the existence of
certain objects in various object sets within the structure. A natural
way would be to generalize the definition of a setoid and add these
components to the structure. However, we found this to be difficult
since it is not clear how to capture the coherence condition which
basically says that any two morphisms which just represent identities
in the strict case should be equal. Instead we will follow a different
approach which can be compared to the definition of environment models
for the $\lambda$-calculus: we shall define a syntax for weak
$\omega$-groupoids and then define a weak $\omega$-groupoid as a
globular set in which this syntax can be interpreted.

\subsection{The syntactical framework}
\label{sec:syntactical-framework}

We start by presenting a syntactical framework which is a syntax for
globular sets. This syntax could be used to identify any globular set
with structure (e.g. weak or strict $\omega$-categories), the specific
aspects of a weak $\omega$-groupoid will be introduced later by adding
additional syntax for objects and other auxilliary syntactic
components. 

Our framework consists of the following components which are defined
by mutual induction 
\footnote{This is an instance of an inductive-inductive
definition in Type Theory, see \cite{SetzerForsberg,AltenkirchEtAl}.}
\begin{description}
\item[Contexts] 
\[\mathsf{Con} : \mathsf{Set}\]
Contexts serve to formalize the existence 
of hypothetical objects which are specified by the category in which
they live. E.g. to formalize ordinary composition we have to assume that 
objects $a,b,c$ and objects in the hom categories $f : a \to b$ and $g
: b \to c$ exist to be able to form $g \circ f : a \to c$.
\item[Categories] 
\[
\AxiomC{$\Gamma : \mathsf{Con}$}
\UnaryInfC{$\mathsf{Cat}~\Gamma : \mathsf{Set}$}
\DisplayProof
\]
Categories are basically paths within a globular set. The set of
expressions for categories depends on the context, e.g. we need at
least to assume that there is one object in the top-level category to
be able to form any other categories.
\item[Objects] 
\[
\AxiomC{$C : \mathsf{Cat}~\Gamma$}
\UnaryInfC{$\mathsf{Obj}~C:\mathsf{Set}$}
\DisplayProof
\]
Given a category we define expressions which identify objects. 
As indicated above this is the main focus of the forthcoming
sections. However, a generic instance of objects are variables which
correspond to projections out of contexts.
\[
\AxiomC{$C : \mathsf{Cat}~\Gamma$}
\UnaryInfC{$\mathsf{Var}~C : \mathsf{Set}$}
\DisplayProof
\]
\end{description}
We now specify the constructors for the various sets (apart form
objects). We use unnamed variables ala deBruijn, hence contexts are
basically a sequence of categories. However, note that this is a
dependent context since the well-formedness of a category expression
may depend on the previous context.
\[
\AxiomC{\mathstrut}
\UnaryInfC{$\varepsilon : \mathsf{Con}$}
\DisplayProof
\qquad
\AxiomC{$C : \mathsf{Cat}~\Gamma$}
\UnaryInfC{$(\Gamma , C) : \mathsf{Con}$}
\DisplayProof
\]
There are two ways to form category expressions: there is the top
level category dentoted by $\bullet$ and given any two objects 
$a,b$ in a category $C$ we can form the hom category $C[a,b]$.
\[
\AxiomC{$\phantom{\Gamma}$}
\UnaryInfC{$\bullet : \mathsf{Cat}~\Gamma$}
\DisplayProof
\]\[
\AxiomC{$C : \mathsf{Cat}~\Gamma\quad a,\,b : \mathsf{Obj}~C$}
\UnaryInfC{$C[ \,a\,,\,b\,] : \mathsf{Cat}~\Gamma$}
\DisplayProof
\]
Using De Bruijn indexing we introduce nameless variables in
contexts as follows:
\[
\AxiomC{$\phantom{I}$}
\dblline
\UnaryInfC{$\mathsf{vz}:\mathsf{Var}~(\mathsf{wk}~C)$}
\DisplayProof
\]
\[
\AxiomC{$v : \mathsf{Var}~C$}
\dblline
\UnaryInfC{$\mathsf{vs}~v : \mathsf{Var}~(\mathsf{wk}~C~D)$}
\DisplayProof
\]
where $\mathsf{wk}$ is weakening defined for categories by
induction on the structure in the obvious way: 
\[
\AxiomC{$C,\,D : \mathsf{Cat}~\Gamma$}
\UnaryInfC{$\mathsf{wk}~C~D : \mathsf{Cat}~(\Gamma,D)$}
\DisplayProof
\]\[
\begin{array}{l}
\mathsf{wk}~\bullet~D\,=\,\bullet\\
\mathsf{wk}~(\homcat{C}{a}{b})~D\,=\,(\mathsf{wk}~C~D)[\mathsf{wk}~a~D,\mathsf{wk}~b~D]
\end{array}
\]
from weakening for objects:
\[
\AxiomC{$x : \mathsf{Obj}~C$}
\AxiomC{$D:\Cat~\Gamma$}
\BinaryInfC{$\mathsf{wk}~x~D:\mathsf{Obj}~(\mathsf{wk}~C~D)$}
\DisplayProof
\]
which is a constructor of $\mathsf{Obj}$. 

Variables become objects by the following constructor of $\Obj$:
\[
\AxiomC{$v : \mathsf{Var}~C$}
\dblline
\UnaryInfC{$\mathsf{var}~v : \mathsf{Obj}~C$}
\DisplayProof
\]


We use the usual arrow notation for categories and objects. For
instance, $\bullet[a,b]$, $\bullet[a,b][f,g]$ and $\alpha :
\Obj~(\bullet[a,b][f,g])$ are pictured respectively as follows:
\[\bfig
\morphism/{}/<300,0>[a`b;]
\efig
\quad\qquad 
\bfig
\morphism/{@{>}@/^1em/}/[a`b;f]
\morphism|b|/{@{>}@/_1em/}/[a`b;g]
\efig
\qquad 
\bfig
\morphism/{@{>}@/^1em/}/[a`b;f]
\morphism|b|/{@{>}@/_1em/}/[a`b;g]
\morphism(250,80)|r|<0,-140>[`;\alpha]
\efig
\]
%
We also write, as usual, $x : a_n\longrightarrow b_n : \cdots
: a_0 \longrightarrow b_0$ for an 
$x : \Obj~(\bullet[a_0,b_0]\cdots[a_n,b_n])$. 


% This is the basic setup for the syntax of weak omega categories,
% obviously more constructors for $\mathsf{Obj}$ are needed which we
% will discussed in the rest of the text. 

\subsection{Interpretation}
\label{sec:interpretation}
Given a globular set we define what we mean by an interpretation of
the syntax. Once we have specified all the constructors for objects a
weak $\omega$-groupoid is given by such an interpretation. The
interpretation of the structural components given in the present
section is fiexd. Again this is reminiscent of enviroment models.

An \emph{interpretation} in a globular set $G:\Glob$ is given by the
following data:
\begin{enumerate}
\item An assignment of sets to contexts:
\[
%<<<<<<< HEAD
\AxiomC{$o : \Obj~C\qquad x : \intpr{\Gamma}$}
\UnaryInfC{$\intpr{o}~x : \mathsf{obj}~(\intpr{C}~x)$}
%=======
%\AxiomC{$\Gamma : \Con$}
%\UnaryInfC{$\intpr{\Gamma} : \Set$}
%>>>>>>> 341cc994e3d493b5de0c47333fc330bf43f3a890
\DisplayProof
\]
\item An assignment of globular sets to category expressions:
\[
\AxiomC{$C : \Cat ~ \Gamma$}
\AxiomC{$\gamma : \intpr{\Gamma}$}
\BinaryInfC{$\intpr{C}~\gamma : \Glob$}
\DisplayProof
\]
\item An assignements of elements of object sets to object
  expressions and variables
\[
\AxiomC{$C : \Cat ~ \Gamma$}
\AxiomC{$A : \mathsf{Obj}~C$}
\AxiomC{$\gamma : \intpr{\Gamma}$}
\TrinaryInfC{$\intpr{A}~\gamma : \obj_{\intpr{C}~\gamma}$}
\DisplayProof
\]
\[
\AxiomC{$C : \Cat ~ \Gamma$}
\AxiomC{$x : \mathsf{Var}~C$}
\AxiomC{$\gamma : \intpr{\Gamma}$}
\TrinaryInfC{$\intpr{x}~\gamma : \obj_{\intpr{C}~\gamma}$}
\DisplayProof
\]
\end{enumerate}
subject to the following conditions:
\begin{eqnarray*}
\intpr{\varepsilon}  & = & 1\\
\intpr{\Gamma , C} & = & \Sigma \gamma : \intpr{\Gamma},\intpr{C} ~
\gamma\\
\intpr{\bullet}~\gamma & = & G\\
\intpr{\homcat{C}{a}{b}}~\gamma & = & \mathsf{hom}_{\intpr{C}
  \gamma}~(\intpr{a} ~ \gamma)~(\intpr{b}~\gamma)\\
\intpr{\mathsf{var}~x}~\gamma  & = & \intpr{x}~\gamma \\
\intpr{\wk~x}~(\gamma,x)  & = & \intpr{x}~\gamma\\
\intpr{\mathsf{vz}}~(\gamma,a)  & = & a \\
\intpr{\mathsf{vs}\,x}~(\gamma,a)  & = & \intpr{x}~\gamma
\end{eqnarray*}


%%% Local Variables: 
%%% mode: latex
%%% TeX-master: "weakomega2"
%%% End: 
