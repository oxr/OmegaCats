\section{Structure}
\label{sec:structure}



A category, strict or weak, is a globular set with additional
structure. The difference between the strict and the weak case is whether
we adorn the structure with (equational) constraints or whether one instead
of axioms introduces more structure, which witnesses rather than postulates
the constraints; so-called \emph{coherence cells}. In this section we
introduce the syntax for the structure of composition and units
giving rise to syntax for what one could call a \emph{pre-monoidal
  globular category}, where composites and units are expressible but
unconstrained by coherence cells.

\subsection{Composition}\label{sec:composition}
%
\newcommand{\cC}{\mathcal{C}}
%
In the ordinary case, a category, $\mathcal{C}$, defines an indexed
operation of composition on its arrows. Explicitly, for $a$, $b$, $c$
objects of $\mathcal{C}$, $f$ in $\homcat{\mathcal{C}}{a}{b}$, $g$  in
$\homcat{\mathcal{C}}{b}{c}$, there is a $gf$ in $\homcat{\mathcal{C}}{a}{c}$.
%
In the higher-dimensional case, $\cC(a,b)$ and $\cC(b,c)$ are not mere
sets but $\omega$-categories and composition extends from sets the
whole hom-categories. Informally: for $a$, $b$, $c$ as before, $f$,
$g$, $n$-cells of homcategories $\cC(a,b)$ and $\cC(b,c)$,
respectively, one requires an $n$-cell $g\circ f$ of $\cC(a,c)$. The
fact that both $f$ and $g$ are of the same relative depth with respect
to $\cC$ is important, as well the fact the homcategories of $f$ and $g$
meet at a common object, $b$, of $\cC$. Following are some
examples of valid compositions for increasing $n$:
\begin{align}
\label{eq:comp1}
a \to^f b \to^g c
&\quad\mapsto\quad
a\to^{gf}c
&
%\label{eq:comp2}
\bfig
\morphism/{@{>}@/^1em/}/<350,0>[a`b;f]
\morphism|b|/{@{>}@/_1em/}/<350,0>[a`b;f']
\morphism(175,50)|r|<0,-100>[`;\alpha]
\morphism(350,0)/{@{>}@/^1em/}/<350,0>[b`c;g]
\morphism(350,0)|b|/{@{>}@/_1em/}/<350,0>[b`c;g']
\morphism(520,50)|r|<0,-100>[`;\beta]
\efig
&\quad\mapsto\quad
\bfig
\morphism/{@{>}@/^1em/}/<350,0>[a`c;gf]
\morphism|b|/{@{>}@/_1em/}/<350,0>[a`c;g'f']
\morphism(175,50)|r|<0,-100>[`;\beta\alpha]
\efig
% \label{eq:comp3}
% \bfig\scalefactor{1.2}
% \morphism/{@{>}@/^1em/}/[a`b;f]
% \morphism|b|/{@{>}@/_1em/}/[a`b;f']
% \morphism(180,50)/{@{>}@/_.1em/}/<0,-100>[`;\alpha]
% \morphism(320,50)|r|/{@{>}@/^.1em/}/<0,-100>[`;\alpha']
% \morphism(500,0)/{@{>}@/^1em/}/[b`c;g]
% \morphism(500,0)|b|/{@{>}@/_1em/}/[b`c;g']
% \morphism(680,50)/{@{>}@/_.1em/}/<0,-100>[`;\beta]
% \morphism(820,50)|r|/{@{>}@/^.1em/}/<0,-100>[`;\beta']
% \morphism(200,0)<100,0>[`;\gamma]
% \morphism(700,0)<100,0>[`;\delta]
% \efig
% &\quad\mapsto\quad
% \bfig
% \morphism/{@{>}@/^1em/}/[a`c;gf]
% \morphism|b|/{@{>}@/_1em/}/[a`c;g'f']
% \morphism(180,50)/{@{>}@/_.1em/}/<0,-100>[`;\beta\alpha]
% \morphism(320,50)|r|/{@{>}@/^.1em/}/<0,-100>[`;\beta'\hspace{-3pt}\alpha']
% \morphism(200,0)<100,0>[`;\delta\gamma]
% \efig\\
\end{align}
\begin{equation}
\label{eq:comp4}
\bfig\scalefactor{1.2}
\morphism/{@{>}@/^1.7em/}/[a`b;f]
\morphism|m|[a`b;f']
\morphism(180,110)/{@{>}@/_.1em/}/<0,-100>[`;\alpha]
\morphism(320,110)|r|/{@{>}@/^.1em/}/<0,-100>[`;\alpha']
\morphism(200,55)<100,0>[`;\gamma]
%
\morphism|b|/{@{>}@/_1.7em/}/[a`b;f'']
\morphism(180,-10)/{@{>}@/_.1em/}/<0,-100>[`;\beta]
\morphism(320,-10)|r|/{@{>}@/^.1em/}/<0,-100>[`;\beta']
\morphism(200,-55)|b|<100,0>[`;\delta]
\efig
\quad\mapsto\quad
\bfig\scalefactor{2}
\morphism/{@{>}@/^1.3em/}/[a`c;f]
\morphism|b|/{@{>}@/_1.3em/}/[a`c;f'']
\morphism(180,50)/{@{>}@/_.1em/}/<0,-100>[`;\beta\cdot\alpha]
\morphism(320,50)|r|/{@{>}@/^.1em/}/<0,-100>[`;\beta'\cdot\alpha']
\morphism(200,0)<100,0>[`;\delta\cdot\gamma]
\efig
\end{equation}
% 
We formalise this as follows.

\subsubsection{Telescopes}
The type $\Obj : \Cat \to \Set$ represents the set of syntactical objects
lying directly in any category. In order to talk about arbitrary
$n$-cells of a category, for instance to define their compositions, we
must introduce \emph{telescopes}.  Informally, a telescope is a
category in a category.  Formally, telescopes, $\Tel$, are defined
below at the same time as their \emph{concatenation} onto a category,
$\dblplus$, which takes a telescope to a category, and therefore
allows us to put objects into a telescope:
\[
\AxiomC{$C : \Cat~\Gamma$}
\AxiomC{$n : \Nat$}
\BinaryInfC{$\Tel~C~n : \Set$}
\DisplayProof
\quad
\AxiomC{$t : \Tel~C~n$}
\UnaryInfC{$C \dblplus\, t \,:\,
  \Cat~\Gamma$}
\DisplayProof
\]
Telescopes are like categories except that the base case is an
arbitrary category $C$ rather than $\bullet$\,:
\[\AxiomC{$\phantom{C}$}
\UnaryInfC{$\bullet : \Tel~C~0$}
\DisplayProof
\quad
\AxiomC{$t : \Tel~C~n \quad a, b : \Obj\,(\conctel{C}{t})$}
\UnaryInfC{$ \telsuc{t}{a}{b} : \Tel~C~(n+1)$}
\DisplayProof
\qquad
\begin{array}{lcl}
C ~\dblplus ~\bullet &=& C \\
C ~\dblplus ~\telsuc{t}{a}{b}
&=&\homcat{(C \dblplus t)}{a}{b}
\end{array}
\]
%
Here, we call $n$ the \emph{length of $t$}, and we say any $x :
\Obj (\conctel{C}{t})$ to be \emph{of depth $n$}.

%
We say that $t$ \emph{lies in} $C$.
%
Note that only the left associative reading of $\dblplus$ makes sense
so expressions like $C \dblplus\, t \dblplus\, u$ are unambiguous.

We say that an object $x : \Obj~(\conctel{C}{t})$ \emph{lies in (the
  telescope) $t$}. When $t$ lies in $C$, $x$ is
called an \emph{object relative to $C$.} 
%
Alternatively, when the category $t$ lies in is not important we use
the following syntactical shorthand:
\[
\AxiomC{$C : \Cat~\Gamma \quad t : \Tel~C~n$}
\UnaryInfC{$\cat{t}\, : \Cat~\Gamma \qquad \cat{t}\,=\,\conctel{C}{t}$}
\DisplayProof
\]

For example, given the category 
$\bfig
\morphism(0,1000)/{@{>}@/^2em/}/<1000,0>[a`b;f]
\morphism(0,1000)|b|/{@{>}@/_2em/}/<1000,0>[a`b;g]
\morphism(350,1150)|l|/{@{>}@/_.5em/}/<0,-300>[`;\varphi]
\morphism(650,1150)|r|/{@{>}@/^.5em/}/<0,-300>[`;\gamma]
\efig
$, one has:
\[\cat{\telsuc{\telzero}{\varphi}{\gamma}}~ = ~
\conctel{\homcat{\homcat{\bullet}{a}{b}}{f}{g}}{\telsuc{\telzero}{\varphi}{\gamma}} ~ = ~ \homcat{\homcat{\homcat{\bullet}{a}{b}}{f}{g}}{\varphi}{\gamma}\enspace.\]
%




\subsubsection{Back to composition}
We use telescopes to define syntax for all compositions of an
$\omega$-category. These are defined mutually recursively with
their extensions to telescopes:
%
\[
\AxiomC{$t : \Tel~(\homcat{C}{a}{b})~n \qquad u :
  \Tel~(\homcat{C}{b}{c})~n$}
\UnaryInfC{$u \circ  t  : \Tel~(\homcat{C}{a}{c})$}
\DisplayProof
\quad
\AxiomC{$
\alpha : \Obj (\homcat{C}{a}{b} \dblplus t)
\qquad 
\beta : \Obj (\homcat{C}{b}{c} \dblplus u)
$}
\UnaryInfC{$\beta\circ \alpha : \Obj (\homcat{C}{a}{c} \dblplus (u \circ t))$}
\DisplayProof
\]
Where $\circ$ is a new constructor of $\Obj$ and $\circ$ for telescopes 
is a function defined by cases
\[\begin{array}{lcl}
\telzero \circ \telzero & = &\telzero\\
\telsuc{u}{a'}{b'} \circ \telsuc{t}{a}{b} & = &\telsuc{(u\circ t) }{a' \circ a}{b' \circ b}
\end{array}\]
% 
Any $\alpha$ and $\beta$ as above are said to be \emph{composable}.
Note that for a fixed category $C$, $\circ$ always defines the
composition in $C$, called \emph{horizontal} in the 2-categorical
case, which can be applied to all composable $(n+1)$-cells of $C$, where
$n$ is the length of the telescopes $t$ and $u$. To compose cells
``vertically'', one moves to a homcategory. In
2-category theory, horizontal composition is usually denoted $\circ$ or
$\ast$ or is left out, whereas vertical composition by
$\cdot\,$. In our case, we always use $\circ$ and
the level we mean is contained in the (implicit) parameter  $C$. For
example, examples in \eqref{eq:comp1} are both horizontal
compositions where $C = \bullet$, while \eqref{eq:comp4} is a
vertical composition where $C = \homcat{\bullet}{a}{b}$. 



\subsection{Units}\label{sec:units}
We generate all higher units from a single constructor $\id$ defined as follows:
\[
\AxiomC{$a : \Obj~C$}
\UnaryInfC{$\id~a : \Obj ~\homcat{C}{a}{a}$}
\DisplayProof
\]
%
By iteration we obtain the unit for horizontal composition of $n$-cells:
\[
\AxiomC{$a : \Obj~C$}
\AxiomC{$ n : \Nat$}
\BinaryInfC{$ \idTel{a}{n} : \Tel ~C~n \qquad \id^n a : \Obj~ (\idTel{a}{n} \Downarrow) $}
\DisplayProof
\]
Again, an iterated unit is defined at the same time as its
telescope.
\[\begin{array}{lclclcl}
\idTel{a}{0} &= &\telzero &\qquad&\id^0 a & =& a\\
\idTel{a}{(n+1)}& = &\telsuc{(\idTel{a}{n})}{\id^n a}{\id^n a}&&\id^{(n+1)}a&= &\id\, (\id^n a)
\end{array}
\]



%%% Local Variables: 
%%% mode: latex
%%% TeX-master: "weakomega-csl"
%%% End: 
