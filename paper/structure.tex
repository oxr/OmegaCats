\section{Structure \textit{(oxr)}}
\label{sec:structure}

\begin{quote}
  \begin{itemize}
  \item composition

  \item identity

  \item inverses

  \end{itemize}
\end{quote}

A category, strict or weak, is a globular set with additional
structure. The difference between the strict and weak case is weather
we adorn the structure with (equational) constraints or whether one instead
of axioms introduces more structure, which witnesses rather than postulates
the constraints; so-called \emph{coherence cells}. In this section we
introduce the (syntax for the) structure of composition and units
giving rise to syntax for what one could call a \emph{pre-monoidal
  globular category}, where composites and units are expressible but
unconstrained by any coherence cells.

\subsection{Composition}\label{sec:composition}
%
\newcommand{\cC}{\mathcal{C}}
%
In the ordinary case, a category, $\mathcal{C}$, defines a partial
operation of composition on its arrows. Explicitly, for $a$, $b$, $c$
objects of $\mathcal{C}$, $f$ in $\homcat{\mathcal{C}}{a}{b}$, $g$  in
$\homcat{\mathcal{C}}{b}{c}$, there is a $gf$ in $\homcat{\mathcal{C}}{a}{c}$.
%
In the higher-dimensional case, $\cC(a,b)$ and $\cC(b,c)$ are not mere
sets but $\omega$-categories and composition extends from sets the
whole hom-categories. Informally: for $a$, $b$, $c$ as before, $f$,
$g$, $n$-cells of homcategories $\cC(a,b)$ and $\cC(b,c)$,
respectively, one requires an $n$-cell $g\circ f$ of $\cC(a,c)$. The
fact that both $f$ and $g$ are of the same relative depth with respect
to $\cC$ is important, as well the fact the homcategories $f$ and $g$
lie in meet at a common object, $b$, of $\cC$. Following are some
examples of valid compositions for increasing $n$:
\begin{align}
\label{eq:comp1}
\bfig
\morphism[a`b;f]
\morphism(500,0)[b`c;g]
\efig
&\quad\mapsto\quad
\bfig
\morphism[a`c;gf]
\efig
\\
\label{eq:comp2}
\bfig
\morphism/{@{>}@/^1em/}/[a`b;f]
\morphism|b|/{@{>}@/_1em/}/[a`b;f']
\morphism(250,50)<0,-100>[`;\alpha]
\morphism(500,0)/{@{>}@/^1em/}/[b`c;g]
\morphism(500,0)|b|/{@{>}@/_1em/}/[b`c;g']
\morphism(750,50)<0,-100>[`;\beta]
\efig
&\quad\mapsto\quad
\bfig
\morphism/{@{>}@/^1em/}/[a`c;gf]
\morphism|b|/{@{>}@/_1em/}/[a`c;g'f']
\morphism(250,50)<0,-100>[`;\beta\alpha]
\efig
% \label{eq:comp3}
% \bfig\scalefactor{1.2}
% \morphism/{@{>}@/^1em/}/[a`b;f]
% \morphism|b|/{@{>}@/_1em/}/[a`b;f']
% \morphism(180,50)/{@{>}@/_.1em/}/<0,-100>[`;\alpha]
% \morphism(320,50)|r|/{@{>}@/^.1em/}/<0,-100>[`;\alpha']
% \morphism(500,0)/{@{>}@/^1em/}/[b`c;g]
% \morphism(500,0)|b|/{@{>}@/_1em/}/[b`c;g']
% \morphism(680,50)/{@{>}@/_.1em/}/<0,-100>[`;\beta]
% \morphism(820,50)|r|/{@{>}@/^.1em/}/<0,-100>[`;\beta']
% \morphism(200,0)<100,0>[`;\gamma]
% \morphism(700,0)<100,0>[`;\delta]
% \efig
% &\quad\mapsto\quad
% \bfig
% \morphism/{@{>}@/^1em/}/[a`c;gf]
% \morphism|b|/{@{>}@/_1em/}/[a`c;g'f']
% \morphism(180,50)/{@{>}@/_.1em/}/<0,-100>[`;\beta\alpha]
% \morphism(320,50)|r|/{@{>}@/^.1em/}/<0,-100>[`;\beta'\hspace{-3pt}\alpha']
% \morphism(200,0)<100,0>[`;\delta\gamma]
% \efig\\
\end{align}
\begin{equation}
\label{eq:comp4}
\bfig\scalefactor{1.2}
\morphism/{@{>}@/^1.7em/}/[a`b;f]
\morphism|m|[a`b;f']
\morphism(180,110)/{@{>}@/_.1em/}/<0,-100>[`;\alpha]
\morphism(320,110)|r|/{@{>}@/^.1em/}/<0,-100>[`;\alpha']
\morphism(200,55)<100,0>[`;\gamma]
%
\morphism|b|/{@{>}@/_1.7em/}/[a`b;f'']
\morphism(180,-10)/{@{>}@/_.1em/}/<0,-100>[`;\beta]
\morphism(320,-10)|r|/{@{>}@/^.1em/}/<0,-100>[`;\beta']
\morphism(200,-55)|b|<100,0>[`;\delta]
\efig
\quad\mapsto\quad
\bfig\scalefactor{2}
\morphism/{@{>}@/^1.3em/}/[a`c;f]
\morphism|b|/{@{>}@/_1.3em/}/[a`c;f'']
\morphism(180,50)/{@{>}@/_.1em/}/<0,-100>[`;\beta\cdot\alpha]
\morphism(320,50)|r|/{@{>}@/^.1em/}/<0,-100>[`;\beta'\cdot\alpha']
\morphism(200,0)<100,0>[`;\delta\cdot\gamma]
\efig
\end{equation}
% 
Examples \eqref{eq:comp1},\eqref{eq:comp2} are compositions in
the category $\bullet$, while \eqref{eq:comp4} is a
composition in $\homcat{\bullet}{a}{b}$. We formalise this as follows.

\subsubsection{Telescopes}
Informally, a \emph{telescope} is a category relative to
another category. Formally, telescopes, $\Tel$, are defined below at the same time as a
\emph{normalisation} function $\Downarrow$, which takes a telescope to
a category:
\[
\AxiomC{$C : \Cat~\Gamma$}
\AxiomC{$n : \Nat$}
\BinaryInfC{$\Tel~C~n : \Set$}
\DisplayProof
\quad
\AxiomC{$\{C : \Cat~\Gamma\}$}
\AxiomC{$T : \Tel~C~n$}
\BinaryInfC{$\cat{T}\, : \Cat~\Gamma$}
\DisplayProof
\]
Telescopes are like categories except that the base case is an
arbitrary category $C$ rather than $\bullet$\,:
\[
\AxiomC{$\{C : \Cat~\Gamma\}$}
\dblline
\UnaryInfC{$\telzero{C} : \Tel~C~0$}
\DisplayProof
\quad
\AxiomC{$t : \Tel~C~n$}
\AxiomC{$a~b : \Obj~(\cat{t})$}
\BinaryInfC{$\telsuc{t}{a}{b} : \Tel ~ C ~ (n+1)$}
\DisplayProof
\]
\[
\AxiomC{$\telzero{C}~\{C\} \Downarrow \, = \, C$}
\DisplayProof
\quad
\AxiomC{$\telsuc{t}{a}{b} \Downarrow \, = \, \homcat{(t\Downarrow)}{a}{b}$}
\DisplayProof 
\]
%
We say that an object $x : \Obj~(\cat{t})$ \emph{lies in (the
  telescope) $t$}. When $t$ is relative to a category $C$, $x$ is
called an \emph{object relative to $C$.} 
%
For example, given a category 
\[\bfig
\morphism(0,1000)/{@{>}@/^2em/}/<1000,0>[a`b;f]
\morphism(0,1000)|b|/{@{>}@/_2em/}/<1000,0>[a`b;g]
\morphism(350,1150)|l|/{@{>}@/_.5em/}/<0,-300>[`;\alpha]
\morphism(650,1150)|r|/{@{>}@/^.5em/}/<0,-300>[`;\alpha']
\efig
\enspace,\]
\[\cat{\telsuc{\telzero{\homcat{\homcat{\bullet}{a}{b}}{f}{g}}}{\alpha}{\alpha'}}\quad\equiv\quad\homcat{\homcat{\homcat{\bullet}{a}{b}}{f}{g}}{\alpha}{\alpha'}\enspace.\]
%
So $\Downarrow$ just concatenates a telescope $t : \Tel~C$ to $C$. The
notation, $\Downarrow$, emphasises the category a telescope
defines. However sometimes we are more interested in the \emph{path}
from a category to one of its subcategories. For that reason we define
$\dblplus$ as version of $\Downarrow$ with the implicit category made
explicit:
\[
\AxiomC{$C : \Cat~\Gamma$}
\AxiomC{$t : \Tel~C~n$}
\BinaryInfC{$C {\dblplus}\, t : \Cat~\Gamma \qquad C \dblplus\, t \,=\,
  t \Downarrow$}
\DisplayProof
\]
\mnote{We could make $\dblplus$ into a real concatenation of
  telescopes and connect $\Downarrow$ and $\dblplus$ by lemmas }
\mnote{However, the principal judgement of all this is $\cat{E}
  \equiv \cat{E'}$ for some expressions $E$, $E'$.}
%
Note that only the left associative reading of $\dblplus$ makes sense
so expressions like $C \dblplus\, t \dblplus\, u$ are unambiguous and $\Downarrow$ always extends as far to the left as
sensible. 



\subsubsection{Back to composition}
We use telescopes to define the syntax for all compositions of an
$\omega$-category. Note that these have to be defined mutually recursively with
their extensions to telescopes:
%
\[
\AxiomC{$
\alpha : \Obj (t \Downarrow)
\qquad 
\beta : \Obj (u \Downarrow)
$}
\UnaryInfC{$\beta\circ \alpha : \Obj (u \circ t \Downarrow)$}
\DisplayProof
\]
is  a new constructor of $\Obj$ where 
\[
\AxiomC{$t : \Tel~(\homcat{C}{a}{b})~n \qquad u :
  \Tel~(\homcat{C}{b}{c})~n$}
\UnaryInfC{$u \circ  t  : \Tel~(\homcat{C}{a}{c})$}
\DisplayProof
\]
is a function on telescopes defined by
\[
\begin{array}{l}
\telzero{C} \circ \telzero{C} \,=\,\telzero{C}\\
\telsuc{u}{a'}{b'} \circ \telsuc{t}{a}{b} \,=\, \telsuc{(u
    \circ t) }{a' \circ a}{b' \circ b}
\end{array}
\]

\subsection{Units}\label{sec:units}
Any $n$-cell has $n$ compositions and so it should have $n$ units. We
generate all from a single constructor $\id$ defined as follows:
\[
\AxiomC{$a : \Obj~C$}
\UnaryInfC{$\id~a : \Obj ~\homcat{C}{a}{a}$}
\DisplayProof
\]
%
By iteration we obtain the unit for composition $n$ levels deep.
\[
\AxiomC{$a : \Obj~C$}
\AxiomC{$ n : \Nat$}
\BinaryInfC{$ \idTel{a}{n} : \Tel ~C~n \qquad \id^n a : \Obj~ (\idTel{a}{n} \Downarrow) $}
\DisplayProof
\]
Again, an iterated unit is defined at the same time as the telescope,
$\idTel{}{}$, it lies in. The definition follows:
\[
\AxiomC{$\idTel{a}{0} \,=\, \telzero{C}$}
\DisplayProof
\AxiomC{$\idTel{a}{(n+1)}\, = \,
  \telsuc{(\idTel{a}{n})}{\id^n a}{\id^n a}$}
\DisplayProof
\]
\[
\AxiomC{$\id^0 a \, =\, a$}
\DisplayProof
\quad 
\AxiomC{$\id^{(n+1)}a\,=\, \id\, (\id^n a)$}
\DisplayProof
\]




%%% Local Variables: 
%%% mode: latex
%%% TeX-master: "weakomega2"
%%% End: 
